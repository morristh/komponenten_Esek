\begin{center}
    \vspace*{1.5cm}
    {\fontsize{20}{20}\textbf{Teknologvisor}}\\
    \vspace{0.7cm}
    {\fontsize{12}{12}\textit{Om teknologen själv får välja}}
\end{center}
\addtocwithheader{Teknologvisor}  % Add entry to TOC and set header\noBackground
\noBackground

\newpage
\resetBackground


\subsection*{Porthos visa} 
\index[alfa]{Porthos visa}
\index[anfa]{Jag vill börja gasqua!}
\songinfo{Mel: You can't get a man with a gun
\\Text: T Andrén}

\begin{parse lines}[\noindent]{#1\\}
    Jag vill börja gasqua!
    Var fan är min flaska?
    Vem i helvete stal min butelj?
    Skall törsten mig tvinga?
    En TT börja svinga?
    Nej för fan bara blunda och svälj!
    Vilken smörja!
    Får jag spörja?
    Vem för fan tror att jag är en älg?
    Till England vi rider,
    och sedan vad det lider,
    träffar vi välan på någon pub
    Och där skall vi festa!
    Blott dricka utav det bästa
    utav Whiskey och Portvin
    Jag tänker gå hårt in
    för att prova på rubb och stubb
\end{parse lines}

\newpage


\subsection*{En komplex värld} 
\index[alfa]{En komplex värld}
\index[anfa]{Alla jävla bevis}
\songinfo{Mel: En helt ny värld (ur Aladdin)\\
Text: Ellinor Persson F07 och Andreas Tågerud F06
}

\begin{parse lines}[\noindent]{#1\\}
    Alla jävla bevis
    inses lätt som en övning.
    Javakursen en prövning
    för min bristande logik

    Ska man komma ihåg
    alla formler i huvet?
    Formelsamlingen, du vet,
    säger inget om det här

    En komplex värld
    Vad fan betyder bijektiv?
    Ingenting stämmer här, där allt jag lär,
    blir glömt snart efter tentan
    Hur ska det gå?
    Och det är bara vecka två
    Känner en underton av aggression
    mot allt Sven Spanne skrivit i sin bok

    \textit{Jag kan transponera den!}

    Jag kan lära dig C
    Matematiska under
    Oförglömliga stunder
    när vi tentar mekanik
\end{parse lines}

\vissteduatt{Visste du att Sven Spanne har skrivit kursboken som används i
\\System och transformer?}
\newpage

\begin{parse lines}[\noindent]{#1\\}
    Det ska nog gå
    Det sa din mamma med igår
    All tid tillvaratas, jag är i fas,
    och bor i mattehuset
    Nu är jag lärd
    Till denna svåra ekvation
    jag på frekvenssidan en lösning fann,
    den låg där i en helt ny värld: Laplace

\end{parse lines}


\subsection*{SI - Système International d'Unités} 
\index[alfa]{SI - Système International d'Unités}
\index[anfa]{1, 2, 75, 6, 7}
\songinfo{Mel: Studentsången}

\begin{parse lines}[\noindent]{#1\\}
    W kg m Wb s
    Ω m T A rad
    cd Sv N s
    Ω A m lx dB
    °C W/m²
    J/kg H V C
    kg/m3 mol
    m/s²
    m/s²
    F !
\end{parse lines}

\vissteduatt{Visste du att SI-låttexten finns i TeFyMa?}
\newpage

\subsection*{Man ska ha MATLAB} 
\index[alfa]{Man ska ha MATLAB}
\index[anfa]{Man ska ha MATLAB}
\songinfo{Mel: Man ska ha husvagn}

\begin{parse lines}[\noindent]{#1\\}
    Jag har prövat nästan allt som finns att pröva på
    Beta, kulram, räknesticka, tärning eller så
    Jag har kalkylerat på de konstigaste sätt
    och nu så har jag kommit på hur man ska räkna rätt

    Man ska ha MATLAB - då är kalkylen redan klar
    Man ska ha MATLAB - det har jag sett att andra har
    Man ska ha MATLAB - det är min livsfilosofi
    Man ska ha MATLAB - för då blir man fri

    I många år så var jag inte alls så särskilt lärd
    Jag visste ej vad som vänta' mig i denna stora värld
    Men sen kom jag till LTH, och ända sedan dess
    så har jag funnit livets stora lyxdelikatess

    Man ska ha MATLAB - så man slipper tänka alls
    Man ska ha MATLAB - ja, då går allting som en vals
    Man ska ha MATLAB - det bygger på nån slags logik
    Man ska ha MATLAB - för då blir man rik

    5 minuter mekanik och 5 minuter statfys
    5 minuter plottande och 5 minuter analys
    5 minuter fråga phadder, 5 minuter stopp
    5 minuter tänka själv och sen så ger man opp
\end{parse lines}

\vissteduatt{Vad kom först MATLAB eller Maple?}

\newpage

\begin{parse lines}[\noindent]{#1\\}
    Man ska ha MATLAB - och datasalens friska luft
    Man ska ha MATLAB - det tycker tjejerna är tufft
    Man ska ha MATLAB - när ryssen kommer med sitt MIG
    Man ska ha MATLAB - då vinner man i krig!

\end{parse lines}

\subsection*{Tenta efter jul} 
\index[alfa]{Tenta efter jul}
\index[anfa]{Tenta efter jul}
\songinfo{Mel: Mössens julafton\\
D-sektionen sångarstriden 2017}

\begin{parse lines}[\noindent]{#1\\}
    När julen börjar närmas
    och man vill koppla av
    Så kommer tentaplugget
    här och ställer sina krav

    Jag börjar kompromissa
    gör julrimmen i C
    Försöker strukturera
    pluggar fram till klockan tre

    Programmering lin-jär algebra
    Endim å reglerteknik är ingenting att ha
    Tenta efter nyår är ju trist
    Men skippar man för många ja då blir man alkolist

    Skål! 
\end{parse lines}


\newpage
\noBackground

\begin{textblock*}{2.7cm}(5.5cm,3.8cm) % {width}(x, y)
    \includegraphics[width=4.3cm]{./bilder/Lågpassfilter.png}
\end{textblock*}

\subsection*{Teknologvisa} 
\index[alfa]{Teknologvisa}
\index[anfa]{Jag är teknolog och helt OK}
\songinfo{Mel: The Lumberjack Song (Monty Python)\\ Sångarstriden 1982\\ Kursivt sjunges av sångförman}

\noindent\textit{Jag är teknolog och helt OK\\
Jag jobbar hårt och jag roar mig}\\\\
\noindent Han är teknolog och helt OK\\
Han jobbar hårt och han roar sig\\\\
\noindent\textit{Teknik är ball\\
Jag kan Pascal\\
Till Lophtet vill jag gå\\
Där träffas alla vänner\\
som är från LTH}\\\\
\noindent Teknik är ball\\
Han kan Pascal\\
Till Lophtet vill han gå\\
Där träffas alla vänner\\
som är från LTH\\\\
\noindent För han är teknolog och helt OK\\
Han jobbar hårt och han roar sig\\\\

\vissteduatt{Visste du att "Han" kan eneklt bytas ut mot "Hon" eller "Hen"!}

\newpage
\resetBackground

\noindent\textit{Min mattebok \\
den gör mig klok\\
Jag läser kärnfysik\\
Jag går på föreläsning\\
och älskar juridik}\\\\
\noindent Hans mattebok\\
den gör han klok\\
Han läser kärnfysik\\
Han går på föreläsning\\
och älskar juridik???\\\\
\noindent Men han är teknolog och helt OK\\
Han jobbar hårt och han roar sig\\\\
\noindent\textit{Som ekonom jag blir fantom\\
Konkurser gör mig säll\\
Till flickor blankt jag nekar\\
Jag älskar en tabell}\\\\
\noindent Som ekonom han blir fantom???\\
konkurser...\\
Nää, BUU!!\\\\
\noindent Men han är teknolog och helt OK\\
Han jobbar hårt och han roar sig\\


\newpage
\enlargethispage{2cm}

\subsection*{\colorbox{yellow}{Glädjekällan}}
\index[alfa]{?}
\index[anfa]{?}
\songinfo{Mel: Pärleporten\\ 
Text: Gabriel Andersson E21 och \colorbox{yellow}{Torgil Mitja E22}\\
E-sektionen Sångarstriden 2023}

\begin{parse lines}[\noindent]{#1\\}
    Som en ljuvlig glädjekälla
    Ljuv och god, ölen e
    I mig, den hastigt jag hälla
    Två fingrar ser nu ut som tre  

    Jag förade för mycke
    Varannan vatten glömde jag
    Sluddrar som en gammal jycke
    Färden mot gasque blir svår idag
   
    Kön till sittningen går långsamt        
    Kön till baren likaså
    Barstoppad blir jag, så pinsamt        
    Säg får man punsch i glaset då
    
    Jag för spriten fattat tycke
    Så jag däckar på vårt bord
    Tröjan den av mig jag ryckte
    Sexmästaren har nu fått nog
    
    Under över alla under
    Vaknar upp i ett buskage
    Bee-ger mig mot Lophtets dunder
    Trillar från kullen med kurage

    Jag dansar helt för mycke
    Glömmer Newtons fjärde lag
    Den som vatten ej samtycker
    Får känn av tinnitus nästa dag
    \end{parse lines}

\newpage

\subsection*{O, hemska labb} 
\index[alfa]{O, hemska labb}
\index[anfa]{O, hemska labb}
\songinfo{Mel: O, helga natt}

\begin{parse lines}[\noindent]{#1\\}
    O, hemska labb, o grymma kval imorgon
    Här sitter jag och förstår ingenting
    Hela mitt inre är fyllt utav ett motstånd
    emot eländig elektrisk mätteknik
    Jag skulle nog behöva lite ledning,
    här räcker inte min kapacitans
    Kondensatorer och felvända dioder
    O, hemska labb nu vill jag koppla af
    O, hemska labb ty detta blir min graf

    O, hemska labb, o grymma kval imorgon
    Här sitter jag och förstår ingenting
    Hela programmet är fyllt utav funktioner
    som innehåller en himla massa fel
    Pekare som inte har nån riktning,
    oändliga loopar, oj vad jag blir sträng!
    Åh, kompilera, hur ska det här fungera?
    O, hemska labb, nu vill jag logga ut
    O, hemska labb, ty detta blir mitt slut
\end{parse lines}

\newpage