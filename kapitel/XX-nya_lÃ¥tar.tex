\textbf{oklart kapitel}


\newpage


\subsection*{Temperaturen} 
\index[alfa]{Temperaturen}
\index[anfa]{Temperaturen}
\songinfo{Text: Mora Träsk}

\begin{parse lines}[\noindent]{#1\\}
    När temperaturen är hög uti kroppen
    närmare 40 än trettiosju komma fem
    så ska det vara när ångan är uppe
    och så är fallet i detta nu

    Vi rullar, vi rullar…
\end{parse lines}

\colorbox{orange}{ska vi ha de andra verserna?}

\begin{parse lines}[\noindent]{#1\\}
    Nu får för varandra vi oss presentera
    och ifrån stolen vi reser oss opp.
    Herrarna börja och djupt de sig buga
    damerna sedan från stolen far opp.

    August och Lotta…

    Rakt över bordet hittar vi en granne
    ett stadigt tag i hans händer vi ta.
    Sen ska vi sjunga som båten på havet
    glasen vi akta så de ej gå i kras.

    Det rullar, det rullar…

    Ja, ikväll ska vi alla vara glada och sjunga
    glädjen skall vara överst på topp.
    Jenka vi dansar så plankorna gungar
    om någon trillar vi reser han opp.

    Det rullar, det rullar…
\end{parse lines}

\vissteduatt{Visste du att...}


\newpage


\subsection*{Flerdimensionell ångest} 
\index[alfa]{Flerdimensionell ångest}
\index[anfa]{Flerdimensionell ångest}
\songinfo{Mel: Härjarevisan\\
Text: Daniel Milve F11}

\begin{parse lines}[\noindent]{#1\\}
    Jag har aldrig sett mig själv som välkalkylerad
    Riktningsderivata gör min hjärna punkterad
    Man bordet börjat lyssna redan
    Typ när nollningen tog stopp
    Än har jag ingen aning om vad Green’s formel säger
    Dock vet jag klart och tydligt vad de dryga förtäljer:
    Skillnaden mellan kurs och fest är 
    Kurser kan man göra om!

    Men så nu ska vi ut och tenta
    Statens små bidrag hämta
    Differentialer, integral och vektoranalys (i planet!)
    Fem timmar med hårda tag och
    Nu änteligen lär jag
    Kunna dra nån nytta av vad Månsson sade vecka två!
\end{parse lines}

\vissteduatt{Visste du att...}


\newpage


\subsection*{Ordenskapitlet} 
\index[alfa]{Ordenskapitlet}
\index[anfa]{Ordenskapitlet}
\songinfo{Mel: Vårvindar friska\\
Lundakarnevalen 1998}

\begin{parse lines}[\noindent]{#1\\}
    Hemliga riter
    Sprit, flera liter
    Nu skall ni vigas in i vår sekt
    Blodiga offer
    Otäcka stoffer
    Ordnar uppå suspekt högtidsdräkt
    Var orden har väl sin ritual
    Flertalet har dock gjort detta val:
    Blanda din rit
    Med rikligt med sprit 
    Så orgien blir total!
\end{parse lines}

\vissteduatt{Visste du att...}


\newpage


\subsection*{\colorbox{yellow}{Agentgrogg}} 
\index[alfa]{Agentgrogg}
\index[anfa]{Jag har ett hemligt groggrecept}
\songinfo{Mel: Sjörövar-Fabbe\\
Lundakarnevalen 2006}

\begin{parse lines}[\noindent]{#1\\}
    Jag har ett hemligt groggrecept
    Som jag aldrig någonsin har släppt
    Visst kan du få det utav mig
    Men sen’ måste jag döda dig

    Ta ur frysen is
    Hitta tonic på nå’t vis
    Ta en skvätt med gin
    Och blanda i bägaren din

    Jag har ett hemligt groggrecept
    Som jag aldrig någonsin har släppt
    Nu har du fått det utav mig
    Så nu måste jag döda dig
    SKÅL!
\end{parse lines}

\vissteduatt{Visste du att...}


\newpage


\subsection*{Dualistisk kanon} 
\index[alfa]{Dualistisk kanon}
\index[anfa]{Dualistisk kanon}
\songinfo{Mel: Broder Jakob\\
Lundakarnevalen 2006}

\noindent \textit{Alkohol:}\\
\begin{parse lines}[\noindent]{#1\\}
    Fördrink, öl, snaps
    Öl, snaps, vin, vin
    Plunta, punsch
    Grogg, shot, shot
    På det jag skylla
    Morgonens bakfylla
    Vi som tål
    Säger skål!
\end{parse lines}

\noindent \textit{Alkoholfritt:}\\
\begin{parse lines}[\noindent]{#1\\}
    Pepsi, Fanta
    Vatten, Pommac
    Tonic, Sprite
    Cola light
    Imorrn’ när ni är tärda
    Tränar jag på Gerda
    Tar det kallt
    Jag minns ALLT!
\end{parse lines}

\vissteduatt{Visste du att...}


\newpage


\subsection*{Vätskeersättning} 
\index[alfa]{Vätskeersättning}
\index[anfa]{Vätskeersättning}
\songinfo{Mel: Sån't är livet\\
Lundakarnevalen 2014}

\colorbox{yellow}{OSÄKER PÅ DENNA}

\begin{parse lines}[\noindent]{#1\\}
    Vätskeersättning
    Vätskeersättning
    Du är en blandning
    Av sött och salt
    Du utelämnar
    Ångest och smärta
    Med dig i kroppen 
    Klarar jag allt

    Ååh resorben
    Bakistabletten
    Ni är mitt skyddsnät
    när jag är svag
    Så fyll nu glasen
    och glöm all oro
    Nu finns det hopp om
    en morgondag!
\end{parse lines}

\vissteduatt{Visste du att...}


\newpage


\subsection*{Baklängesfyllan} 
\index[alfa]{Baklängesfyllan}
\index[anfa]{Baklängesfyllan}
\songinfo{Mel: Rövarnas visa\\
Lundakarnevalen 2018}

\begin{parse lines}[\noindent]{#1\\}
    Vätskeersättning
    Vätskeersättning
    Du är en blandning
    Av sött och salt
    Du utelämnar
    Ångest och smärta
    Med dig i kroppen 
    Klarar jag allt

    Ååh resorben
    Bakistabletten
    Ni är mitt skyddsnät
    när jag är svag
    Så fyll nu glasen
    och glöm all oro
    Nu finns det hopp om
    en morgondag!
\end{parse lines}

\vissteduatt{Visste du att...}


\newpage


\subsection*{Katastrofalt uppvaknande} 
\index[alfa]{Katastrofalt uppvaknande}
\index[anfa]{Jag vaknade i stolen}
\songinfo{Mel: Stad i ljus\\
Lundakarnevalen 2022}

\begin{parse lines}[\noindent]{#1\\}
    Jag vaknade i stolen, har slumrat till här i mitt rum
    Och allting känns eländigt, mitt bord är fyllt med 
    flaskor - en har runnit ut
    Jag tror jag minns spektaklet och alla dom som 
    förde liv
    Konturen av ett party, jag hade efterfest som växte -
    blev massiv

    Jag är full!
    i ett kaosartat rum
    Ge mig frid
    så jag får somna om
\end{parse lines}

\vissteduatt{Visste du att...}


\newpage


\subsection*{O, hemska labb} 
\index[alfa]{O, hemska labb}
\index[anfa]{O, hemska labb}
\songinfo{Mel: O, helga natt}

\colorbox{orange}{OSÄKER PÅ DENNA}

\begin{parse lines}[\noindent]{#1\\}
    O, hemska labb, o grymma kval imorgon
    Här sitter jag och förstår ingenting
    Hela mitt inre är fyllt utav ett motstånd
    emot eländig elektrisk mätteknik
    Jag skulle nog behöva lite ledning,
    här räcker inte min kapacitans
    Kondensatorer och felvända dioder
    O, hemska labb nu vill jag koppla af
    O, hemska labb ty detta blir min graf

    O, hemska labb, o grymma kval imorgon
    Här sitter jag och förstår ingenting
    Hela programmet är fyllt utav funktioner
    som innehåller en himla massa fel
    Pekare som inte har nån riktning,
    oändliga loopar, oj vad jag blir sträng!
    Åh, kompilera, hur ska det här fungera?
    O, hemska labb, nu vill jag logga ut
    O, hemska labb, ty detta blir mitt slut
\end{parse lines}

\vissteduatt{Visste du att...}


\newpage


\subsection*{Man borde…} 
\index[alfa]{Man borde…}
\index[anfa]{Man borde plugga mera}
\songinfo{Mel: Den blomstertid nu kommer}

\colorbox{yellow}{OSÄKER PÅ DENNA}

\begin{parse lines}[\noindent]{#1\\}
    Man borde plugga mera 
    man borde äta grönt
    Man borde motionera
    att träna lär va skönt

    Man borde sluta röka 
    och sköta om sin kropp
    Man borde borde sluta röka 
    men här har vi gett opp
\end{parse lines}

\vissteduatt{Visste du att...}


\newpage


\subsection*{Bevisskål!} 
\index[alfa]{Bevisskål!}
\index[anfa]{Inför varje mattetenta blir jag frälst!}
\songinfo{Mel: Du kan få min gamla cykel\\
I-sektionen Sångarstriden 2013}

\colorbox{yellow}{OSÄKER PÅ DENNA}

\begin{parse lines}[\noindent]{#1\\}
    Inför varje mattetenta blir jag frälst! 
    Jag får göra det som jag gör allra helst 
    När jag pluggar in bevisen 
    Går det lätt jag har devisen: 
    Ta en sup för varje sats som du dig lärt! 
    Analysens huvudsats och konvex mängd (Ta en sup!) 
    Texasgren och bisektrisen, ta en sup! (Ta en sup!) 
    Jordans lemma är helt given och Greens formel redan skriven Cauchy-Riemanns är angiven Ta en sup! 
    Sinussatsen och invers Laplacetransform (Ta en sup!) 
    Satser och bevis gör fyllan helt enorm (Ta en sup!) 
    Borde sluta memorera Men bevisen de blir flera 
    Kedjeregeln ge mig mera Ta en sup! TA DEN NU!
\end{parse lines}

\vissteduatt{Visste du att...}


\newpage


\subsection*{CSN - Din livspartner} 
\index[alfa]{CSN - Din livspartner}
\index[anfa]{Inför varje mattetenta blir jag frälst!}
\songinfo{Mel: Du kan få min gamla cykel\\
Text: Karolina Lindén I11\\
I-sektionen Sångarstriden 2013}

\colorbox{yellow}{OSÄKER PÅ DENNA}

\begin{parse lines}[\noindent]{#1\\}
    Du ska inte tro att du blir nånting,
    ifall inte nån sätter fart
    på studietempot och på din flit,
    så att du blir klar ganska snart
    Jag gör så att matkassan sinar,
    jag gör tentaångesten skör,
    om du inte tar dina tentor,
    så blir det snart vatten och bröd

    Och när du väl tagit examen,
    så väntar jag tåligt på dig,
    med åren och ränta på ränta
    så får du ett liv långt med mig
    Jag tar dina barns julklappspengar,
    din villa får gå på auktion,
    du bor snart på släktingars sängar,
    till sist tar jag barnens pension
\end{parse lines}

\vissteduatt{Visste du att...}


\newpage


\subsection*{CSN-Grinchen} 
\index[alfa]{CSN-Grinchen}
\index[anfa]{Julklappar, glitter och skinka}
\songinfo{Mel: Leende guldbruna ögon\\
Text: Karolina Lindén I11\\
I-sektionen Sångarstriden 2012}

\colorbox{yellow}{OSÄKER PÅ DENNA}

\begin{parse lines}[\noindent]{#1\\}
    Julklappar, glitter och skinka 
    Har jag förälskat mig i 
    Men nu ska tentorna flyttas 
    Så jul kan det aldrig mer bli 
    Jag varit snäll och studerat 
    Sett fram mot tomtens besök 
    Men nu när tentorna flyttas 
    Går hela min dröm upp i rök 
    För CSN vill ej höra 
    Att teknolog firar jul 
    Blott endim, linalg och matstat 
    Ska nu vara allt som är kul
\end{parse lines}

\vissteduatt{Visste du att...}


\newpage


\subsection*{Detta är jag} 
\index[alfa]{Detta är jag}
\index[anfa]{Jag är en enkel mytoman}
\songinfo{Mel: Med en enkel tulipan\\
Lundakarnevalen 2018}

\colorbox{yellow}{OSÄKER PÅ DENNA}

\begin{parse lines}[\noindent]{#1\\}
    Jag är en enkel mytoman
    Som utav ren slentrian
    Han ljugit till mig en plats vid bordet
    och ett glas rödvin
\end{parse lines}

\vissteduatt{Visste du att...}


\newpage


\subsection*{Detta är jag} 
\index[alfa]{Detta är jag}
\index[anfa]{Jag är en enkel mytoman}
\songinfo{Mel: Med en enkel tulipan\\
Lundakarnevalen 2018}

% \colorbox{yellow}{OSÄKER PÅ DENNA}

\begin{parse lines}[\noindent]{#1\\}
    Jag är en enkel mytoman
    Som utav ren slentrian
    Han ljugit till mig en plats vid bordet
    och ett glas rödvin
\end{parse lines}

\vissteduatt{Visste du att...}


\newpage


\subsection*{De som är nyktra} 
\index[alfa]{De som är nyktra}
\index[anfa]{De som är nyktra}
\songinfo{Mel: Du är den ende}

\colorbox{yellow}{OSÄKER PÅ DENNA}

\begin{parse lines}[\noindent]{#1\\}
    De som är nyktra 
    har inte så roli't,
    de har bara ansvar 
    och inte nå't tjolitt-
    an-lej-faderulla 
    men vi som är fulla 
    vi har bara kul nästan jämt

    Det sägs att en män'ska 
    kan va' utan brännvin,
    det stämmer måhända 
    men se blott på den min
    som pryder en absolutist, 
    den e' jävligt trist
    därför så sjunger vi nu:

    De som är nyktra 
    har inte så roli't,
    de har bara ansvar 
    och inte nå't tjolitt-
    an-lej-faderulla 
    men vi som är fulla 
    vi har bara kul nästan jämt
\end{parse lines}

\vissteduatt{Visste du att...}


\newpage


\subsection*{Vi som är nyktra} 
\index[alfa]{Vi som är nyktra}
\index[anfa]{Vi som är nyktra}
\songinfo{Mel: Du är den ende}

\begin{parse lines}[\noindent]{#1\\}
    Vi som är nyktra
    vi har faktiskt roligt.
    Jo visst har vi ansvar,
    men minst lika
    tjolittanlej faderulla
    som ni som är fulla
    som tror ni har kul nästan jämt

    Men tänk då efter
    uppå dagen efter
    de dagar med fester,
    med smärtsamma rester
    utav eran hjärna,
    nog tycker ni gärna.
    Att va nykterist är nåt visst

    Vi som är nyktra,
    vi har bara roligt
    Imorrn kan vi
    återigen ha det
    tjolittanlej faderulla,
    men ni som är fulla
    aj, aj, aj, det är väl för trist
\end{parse lines}

\vissteduatt{Visste du att...}


\newpage


\subsection*{Snapsa mig genom alla åren} 
\index[alfa]{Snapsa mig genom alla åren}
\index[anfa]{Jag ska snapsa mig genom alla åren}
\songinfo{Mel: Genom eld och vatten\\
Text: Tone Riise Åberg, I13\\
I-sektionen Sångarstriden 2014}

\colorbox{orange}{OSÄKER PÅ DENNA}

\begin{parse lines}[\noindent]{#1\\}
    Vi har nya tentor och massa plugg, nöjen de är få. 
    Och examen ligger bortanför vår syn (Längre bort än Vildanden) 
    Fast när faten är tomma men glaset är fullt av någon alkohol, 
    Så tar vi någonting som är mera värt ändå. (mera värt en ett poäng) 

    För vi är teknologer och dricker bara öl eller brännevin förstås 
    Så som punschen till kaffet kan vi lita på 
    Snapsen sviker aldrig oss 

    Jag ska snapsa mig genom alla åren, 
    På sittning och på fest. 
    Svåra kurser nu men slutet ska vi nå. (Eller botten av vårt glas) 
    Det blir Aqua-vit, O-P, Beska o en fläder eller två. 
    Ja, lång tid i Lund men med några snapsar ska det gå! 
    Ja, lång tid i Lund men med några snapsar ska det gå!
\end{parse lines}

\vissteduatt{Visste du att...}


\newpage


\subsection*{Dessertvisan} 
\index[alfa]{Dessertvisan}
\index[anfa]{Gästerna de springer i pausen på dass}
\songinfo{Mel: Sjösala vals}

\colorbox{yellow}{OSÄKER PÅ DENNA}

\begin{parse lines}[\noindent]{#1\\}
    Gästerna de springer i pausen på dass,
    eftersom de fyllt sina magar så pass
    Maten den har smakat,
    jag har ta'tt flera lass,
    men vet att nu nalkas det
    hjortron och glass!

    Kurre, kurre kurre,
    min mage känns så stinn
    Jag ångrar nu så bittert
    att jag har spänt mitt skinn
    Men banta kan jag glömma,
    det är ju finsittning det här!
    Bältet jag spänner upp,
    gör plats för mer dessert!
\end{parse lines}

\vissteduatt{Visste du att...}


\newpage


\subsection*{Tenta efter jul} 
\index[alfa]{Tenta efter jul}
\index[anfa]{Tenta efter jul}
\songinfo{Mel: Mössens julafton\\
Skriva om att D-sektionen vinnande bordvisa (SåS enligt sångarkiv)}

\colorbox{yellow}{OSÄKER PÅ DENNA}

\begin{parse lines}[\noindent]{#1\\}
    När julen börjar närmas
    och man vill koppla av
    Så kommer tentaplugget
    här och ställer sina krav

    Jag börjar kompromissa
    gör julrimmen i C
    Försöker strukturera
    pluggar fram till klockan tre

    Programmering lin-jär algebra
    Endim å reglerteknik är ingenting att ha
    Tenta efter nyår är ju trist
    Men skippar man för många ja då blir man alkolist

    Skål! 
\end{parse lines}

\vissteduatt{Visste du att...}


\newpage


\subsection*{Hacke Hackspett} 
\index[alfa]{Hacke Hackspett}
\index[anfa]{Hacke Hackspett}
\songinfo{Text: Povel Ramel och Georg Eliasson, 1948. \\
Original: "Woody Woodpecker" (Georg F. Tibbles / Ramey Idriss).\\
Povel Ramel och hans glättiga gröngölingar spelade in låten på 78-varvare 23/9 1948.
}

\colorbox{yellow}{OSÄKER PÅ DENNA}

\begin{parse lines}[\noindent]{#1\\}
    Mitt namn är Hacke Hackspett, resande i Schweizerostar!
 
    Hahahahaha! Hahahahaha!  Hör på hackspettens melodi
    Hahahahaha! Hahahahaha!  med sin hackande harmoni!
    
    Han hackar sig fram
    ifrån stam till stam
    och bygger upp sitt höga C
    När du märker hans skratt,
    så ta på dig din hatt
    ty han är på jakt efter tre
    
    Hahahahaha! Hahahahaha!  Har du hört en så’n retfull trall,
    Hahahahaha! Hahahahaha!  när du traskar bland gran och tall
    
    Om hans skönsång nu ej
    gör nå’t intryck på dig,
    alla hackspettars hjärtan slår
    Hahahahaha! Hahahahaha!  varje gång det är sol och vår
    
    Mellanspel!
    
    Hahahahaha! Hahahahaha!  Har du hört en så’n retfull trall,
    Hahahahaha! Hahahahaha!  när du traskar bland gran och tall
    
    Om hans skönsång nu ej
    gör nå’t intryck på dig,
    alla hackspettars hjärtan slår
    Hahahahaha! Hahahahaha!  varje gång det är sol och vår 
\end{parse lines}

\vissteduatt{Visste du att...}


\newpage


\subsection*{Kommunistvikingen} 
\index[alfa]{Kommunistvikingen}
\index[anfa]{Kommunistvikingen}
\songinfo{Mel: When Johnyy comes marchin home\\
Text: Ludwig Linder F19}

\colorbox{yellow}{OSÄKER PÅ DENNA}

\begin{parse lines}[\noindent]{#1\\}
    En viking delar allt den gör, hurra, hurra
    Ordet VIking kommer sig därav, jaha!
    Gammavågor sprids överallt,
    när vikingen sprängt sitt kärnkraftverk,
    då är KGB ute på jakt!

    En viking skjuter rymdraket
    Hurra, hurra!
    Men på månen landar någon fet
    USA, USA!
    Man måste använda all sin list
    för att inte drabbas av näringsbrist
    för våran viking han är kommunist!
\end{parse lines}

\vissteduatt{Visste du att...}


\newpage


\subsection*{Jag är liten nolla} 
\index[alfa]{Jag är liten nolla}
\index[anfa]{Jag är liten nolla}
\songinfo{Mel: Jag är fattig bonddräng\\
Text: JO Sivoft, E94}

\begin{parse lines}[\noindent]{#1\\}
    Jag, en liten nolla, på Elektro jag går
    Dagar går och kommer, medan jag pluggar på
    Labbar, löddar, räknar, programmerar och lär, 
    Går på föreläsning, inför tentan jag svär

    Jag en fattig nolla, pasta lever jag på
    Och när fredan kommer till Edekvata jag gå
    Sen, när jag blitt livad, vill jag dansa, umgås
    Vila hos en flicka, vill jag också förstås

    Sen så kommer helgen, och då vill CSN
    Att jag pluggar satan, men då festar jag än
    40 timmars vecka, gäller inte för oss
    För oss teknologer, e de dubbelt förstås

    Så går hela veckan, varje läsperiod
    Åren går och kommer, men jag är vid gott mod
    Jag tar mina tentor, samlar på mig poäng, 
    Jag tar min examen, sen så blir jag utslängd

    Nu så väntar livet, som civilingenjör
    Nu så ska jag skörda, tjäna pengar som smör
    Men man jobbar sliter, si så där 40 år
    Till barn, familj o staten, alla pengarna går

    Så när dagen kommer, invid himmelens port, 
    Lite rädd och lessen, för de synder jag gjort
    Inte skattefuska, köra fort, supa loss
    Herren Gud i himlen, är väl missnöjd förstas

    Jag, vid pärleporten, blir nu eftertänksam
    De allra bästa åren, alltför snabbt de försvann
    Hade allt för bråttom, bort från de som var bäst
    Åren på Elektro, saknar jag allra mest

    Men då säger Herren: (civil) ingenjören, kom hit! 
    Jag har sett din strävan, och ditt eviga slit
    Därför, ingenjören, är du välkommen här
    Himmelens Elektro till du antagen är

    Jag som liten ängel, står så still inför Gud, 
    och sen klär han på mej en Elektrovit skrud
    Nu du, säger Herren, börjar vi om igen 
    Nu du, liten nolla, nu har du kommit hem

    Till dig liten nolla, sensmoralen den är
    Ha ej allt för bråttom, under tiden du lär
    Tids nog får du jobba, resten utav ditt liv
    Därför ta till vara, på studentlivets tid
\end{parse lines}

\vissteduatt{Visste du att... sjunges gärna på nollegasque(???)}


\newpage


\textbf{Skånska visor}

\newpage


\subsection*{Skåne} 
\index[alfa]{Skåne}
\index[anfa]{Nu har det blivit dags att dricka skåne}
\songinfo{Mel: Lite grann från ovan}

\begin{parse lines}[\noindent]{#1\\}
    Nu har det blivit dags att dricka Skåne
    En liten sträv, så gyllengul som raps
    Det är det bästa mellan sol och måne
    Som lätt får mången stark till snabb kollaps
    Så fatta nu din hand om hela Skåne
    Här kommer södra Sverige i en snaps
    Känn hur Lund och Smygehuk
    slår små volter i din buk
    Aquaviten ska va' gul
    och heta Skåne
\end{parse lines}

\vissteduatt{Visste du att fler skånska snapsvisor finns på sida…}


\newpage


\subsection*{På bordet står en Skåne} 
\index[alfa]{På bordet står en Skåne}
\index[anfa]{Framförmig på bordet står en Skåne}
\songinfo{Mel: Lite grann från ovan}

\colorbox{orange}{OSÄKER PÅ DENNA}

\begin{parse lines}[\noindent]{#1\\}
    Framför mig på bordet står en Skåne
    Den bästa lilla snapsen man kan få
    Jag skulle känna mig liksom en fåne
    Om odrucken den längre skulle stå

    Så alla ni som har nånting i glaset
    Fatta nu så varligt om dess lilla fot
    För det min liv o’ kniv 
    Alla tiders perspektiv
    För din mage att en Skåne ta emot
\end{parse lines}

\vissteduatt{visste du att...}


\newpage


\subsection*{Till den skånska metropolen Vinslöv} 
\index[alfa]{Till den skånska metropolen Vinslöv}
\index[anfa]{Vinslöv}
\songinfo{Mel: Wiensk operettvals}

\colorbox{yellow}{OSÄKER PÅ DENNA}

\begin{parse lines}[\noindent]{#1\\}
    I vin, i vin, i vin, i vin,
    i Vinslöv bor min mor
    På Hven, på Hven, på Hven, på Hven,
    påven han bor i Rom
    I Rom, i Rom, i Rom, i Rom,
    i rompan på en ko
    I ko, i ko, i ko, i ko,
    i Kosta göra man glas
    I glas, i glas, i glas, i glas,
    i glas där har man vin
    I vin, i vin, i vin, i vin,
    i Vinslöv bor min mor
\end{parse lines}

\vissteduatt{visste du att...}


\newpage


\subsection*{Skåne, skåne} 
\index[alfa]{Skåne, skåne}
\index[anfa]{Skåne, alla älskar Skåne}
\songinfo{Mel: Bamse}

\colorbox{orange}{OSÄKER PÅ DENNA}

\begin{parse lines}[\noindent]{#1\\}
    Skåne, Skåne alla älskar Skåne
    Eda päror, sill och ta en snaps
    De e livet när de e mi’sommar
    Höj nu armen o’ ta den i ett nafs
    Skåne, Skåne, Skåne, Skåne, Skåne, Skåne,
    Skåne, Skåne, Skåne, Skåne, Skåne,
    Skåne, Skåne, Skåne, Skål!
\end{parse lines}

\vissteduatt{visste du att...}


\newpage


\subsection*{Lite grann från ovan} 
\index[alfa]{Lite grann från ovan}
\index[anfa]{Lite grann från ovan}
\songinfo{Text: Lasse Dahlqvist\\
Sång: Edvard Persson}

% \colorbox{yellow}{OSÄKER PÅ DENNA}

\begin{parse lines}[\noindent]{#1\\}
    Jag är en liten gåsapåg från Skåne
    En skåning som Ni vet är alltid trygg
    Och fast jag är så nära sol och måne
    Jag sitter säkert på min gåsarygg
    Långt under mig det ligger som en tavla
    Det vackraste i världen man kan se
    Både skogar, sjö och strand
    Blir ett enda sagoland
    När man ser det lite grann så här från ovan

    Där ligger gamla slott och härresäten
    Som minnen från den stolta tid som flytt
    Och aldrig skall den tiden bli förgäten
    Men inget slag skall stånda här på nytt
    Nej, dessa fält skall bära samma skördar
    Som de har gjort i sekelflydda dar
    Ja det är min liv och kniv
    Alla tiders perspektiv
    När man ser det lite grann så här från ovan

    Du kära gås som stolt i skyn dig svingar
    Har ingen farlig last att kasta ned
    Ty du bär fredens vita vackra vingar
    Som världen längtar efter mer och mer
    När människobarnen går därnere och kivas
    Då resonerar du nog liksom jag
    Tänk vad skönt det är ändå
    Att få sväva i det blå
    Och se Er lite grann så här från ovan

    Nu jordens alla murar börjar skaka
    En samling dårar satt vår värld i brann
    Vad fäderna byggt upp blir pannekaka
    Det ryck och slits i gamla vänskapsband
    Ack kära Ni som slåss därner på jorden
    Kom upp och ta en liten titt med mig
    Jag är ganska säker på
    Att Ni skäms en smula då
    När Ni ser vår gamla jord så här från ovan
\end{parse lines}

\vissteduatt{visste du att...}


\newpage

\textbf{Klassiska visor}

\newpage


\subsection*{Sveriges nationalsång / Du gamla, du fria / Nationalsången} 
\index[alfa]{Du gamla du fria}
\index[anfa]{Du gamla du fria}
\songinfo{Text: Richard Dybeck}

\begin{parse lines}[\noindent]{#1\\}
    Du gamla, Du fria, Du fjällhöga Nord,
    du tysta, du glädjerika sköna
    Jag hälsar dig vänaste land uppå jord,
    Din sol, din himmel, dina ängder gröna,
    Din sol, din himmel, dina ängder gröna

    Du tronar på minnen från fornstora dar,
    då ärat ditt namn flög över jorden
    Jag vet att du är och du blir vad du var
    Ja, jag vill leva, jag vill dö i Norden
    Ja, jag vill leva, jag vill dö i Norden
\end{parse lines}

\vissteduatt{visste du att...En händelse som enligt historien bidrog till \\
att sången började få status som nationalsång var vid en promotionsmiddag \\
vid Lunds Universitet våren 1893, då Kung Oscar II ställde sig upp när sången framfördes.}


\newpage


\subsection*{Studentsången} 
\index[alfa]{Studentsången}
\index[anfa]{Sjungom studenten}
\songinfo{Text: Herman Sätherberg\\ 
Musik: Prins Gustaf}

\begin{parse lines}[\noindent]{#1\\}
    Sjungom studentens lyckliga dag,
    låtom oss fröjdas i ungdomens vår!
    Än klappar hjärtat med friska slag,
    och den ljusnande framtid är vår.
    Inga stormar än
    i våra sinnen bo,
    hoppet är vår vän,
    och vi dess löften tro,
    när vi knyta förbund i den lund,
    där de härliga lagrarna gro!
    där de härliga lagrarna gro!
    Hurra!
\end{parse lines}

\vissteduatt{visste du att...}


\newpage


\subsection*{Brevet från kolonien} 
\index[alfa]{Brevet från kolonien}
\index[anfa]{Brevet från kolonien}
\songinfo{Cornelis Vreeswijk}

\colorbox{yellow}{OSÄKER PÅ DENNA}

\begin{parse lines}[\noindent]{#1\\}
    Hejsan morsan, hejsan stabben
    Här är brev från älsklingsgrabben
    Vi har kul på kolonien
    Vi bor tjugoåtta gangstergrabbar i en

    Stor barack med massa sängar
    Kan ni skicka mera pengar?
    För det vore en god gärning
    Jag har spelat bort vartenda dugg på tärning

    Här är roligt vill jag lova
    Fastän lite svårt att sova
    Killen som har sängen över mig
    Han vaknar inte han när han behöver, nej

    Jag har tappat två framtänder
    För jag skulle gå på händer
    När vi lattjade charader
    Så när morsan nu får se mig får hon spader

    Ute i skogen finns baciller
    Men min kompis han har piller
    Som han köpt utav en ful typ
    Och om man äter dem blir man en jättekul typ

    Våran fröken är försvunnen
    Hon har dränkt sig uti brunnen
    För en morgon blev hon galen
    När vi släppte ut en huggorm i matsalen

    Men jag är inte, rädd för spöken
    För min kompis han har kröken
    Som han gjort utav potatis
    Och som han säljer i baracken nästan gratis

    Föreståndaren han har farit
    Han blir aldrig var han varit
    För polisen kom och tog hand
    Om honom förra veckan när vi lekte skogsbrand

    Ute i skogen finns det rådjur
    I baracken finns det smådjur
    Och min bäste kompis Tage
    Han har en liten fickkniv inuti sin mage

    Honom ska de operera
    Ja, nu vet jag inge mera
    Kram och kyss och hjärtligt tack sen
    Men nu ska vi ut och bränna grannbaracken
\end{parse lines}

\vissteduatt{visste du att...}


\newpage


\subsection*{Min ponny - nostalgiska visor??} 
\index[alfa]{Min ponny - nostalgiska visor??}
\index[anfa]{Min ponny - nostalgiska visor??}
\songinfo{Gullan Bornemark}

\colorbox{yellow}{OSÄKER PÅ DENNA}

\begin{parse lines}[\noindent]{#1\\}
    På fyra ben går den som jag gillar allra bäst
    Gillar, gillar, gillar allra bäst
    Jag sitter på hans rygg för han är min lilla häst
    Är min lilla häst
    Vad du är söt min kära lilla ponny
    Vad du är snäll, min kära lilla häst!
    Du säger ingenting, min kära lilla ponny
    Men du är den jag gillar bäst
    Se pälsen den är svart som på kappan på en präst
    Kappan, kappan, kappan på en präst
    En mule mjuk som sammet det har min lilla häst
    Har min lilla häst
    Vad du är söt…
\end{parse lines}

\vissteduatt{visste du att...}


\newpage


\subsection*{Fantomens brallor} 
\index[alfa]{Fantomens brallor}
\index[anfa]{Fantomens brallor}
\songinfo{Lasse Åberg}

\colorbox{yellow}{OSÄKER PÅ DENNA}

\begin{parse lines}[\noindent]{#1\\}
    Ingen har sett Fantomen utan kläder
    Klädd i pyjamas och stövlar av läder
    Han drar nog bort en rand där fram
    När han kissar bakom trädets stam

    Oh, vandrande vålnad, kliar inte sviden
    När du knegar i djungeln hela tiden
    Gör som Guran skaffa dig en kjol
    Det är bättre under Afrikas sol

    Ingen har sett honom kavla upp ärmen
    Ljusblå lekdräkt trots fukten och värmen
    Men han blev nog frusen om sin häck
    Om han satt i grottan alldeles näck

    Oh, vandrande vålnad, kliar inte sviden
    När du knegar i djungeln hela tiden
    Gör som Guran skaffa dig en kjol
    Det är bättre under Afrikas sol

    Fantomen lättar inte på kalsongen
    Nej, han håller värmen stången
    Ibland tar han på sig ännu mer
    När Mr. Walker sig till stan beger

    Oh, vandrande vålnad, kliar inte sviden
    När du knegar i djungeln hela tiden
    Gör som Guran skaffa dig en kjol
    Det är bättre under Afrikas sol

\end{parse lines}

\vissteduatt{visste du att...}


\newpage


\subsection*{Skånska slott och herresäten} 
\index[alfa]{Skånska slott och herresäten}
\index[anfa]{Skånska slott och herresäten}
\songinfo{Text: Hjalmar Gullberg, Bengt Hjelmqvist,
Sång: Edvard Persson}

\colorbox{yellow}{OSÄKER PÅ DENNA}

\begin{parse lines}[\noindent]{#1\\}
    På himmelen vandra sol, stjärnor och måne
    Och kasta sitt fagraste ljus över Skåne
    På höga och låga, på stort och på smått
    På statarens koja och ädlingens slott

    Se månstrålen in genom blyrutan faller
    Och tecknar på golvet det järnsmidda galler
    Stolts jungfrun hon drömmer i majnattens ljus
    Att friare komma till Glimmingehus

    På utflykt till Bokskogen Malmöbon glor upp
    Mot raden av strålande fönster på Torup
    Att smaka på kaka som bakats på spett
    Dig ber hennes nåd, friherrinnan Coyet

    Där rådjuren skymta bak'vitgråa stammar
    Man ser Toppela'gård med broar och dammar
    Systemet på sprit och på skatterna sta'n
    Där lurar belåtet fiskalen Aschan

    Med port genom huset och gamla kanaler
    Lyss Skabersjö ännu till jaktens signaler
    Själv kungen i nåder far dit från sitt slott
    Och skjuter fasaner med grevarna Thott

    Och därefter hälsar han på baron Trolle
    Och jagar och spelar sin sans och sin nolle
    Allt medan baronens gemål
    Plockar gräs åt rastupp och rashöna på Trollenäs
\end{parse lines}

\vissteduatt{visste du att...}


\newpage


\subsection*{Minnet} 
\index[alfa]{Minnet}
\index[anfa]{Jag har tappat mitt minne}
\songinfo{Mel: Memory}

\colorbox{yellow}{OSÄKER PÅ DENNA}

\begin{parse lines}[\noindent]{#1\\}
    Minne, jag har tappat mitt minne, 
    är jag svensk eller finne, kommer inte ihåg

    Inne, är jag ut eller inne, 
    jag har luckor i minnet, 
    såndär små ALKO-HÅL
    Men besinn er, man tätar med det brännvin man får, 
    fastän minnet, och helan går
\end{parse lines}

\vissteduatt{visste du att...}


\newpage

\textbf{Vinvisor}

\newpage


\subsection*{Imsig vimsig} 
\index[alfa]{Imsig vimsig}
\index[anfa]{Imsig vimsig blir man}
\songinfo{Mel: Imse vimse spindel}

\colorbox{yellow}{OSÄKER PÅ DENNA}

\begin{parse lines}[\noindent]{#1\\}
    Åsnan dricker vatten, det gör inte vi
    Vi dricker bara sådant folk har trampat i
    Kamelen uti öknen söker en oas
    Det gör inte vi, vi har vin i våra glas

    Imsig vimsig blir man, utav lite vin
    Kliver upp på stolen, verkar piggelin
    Ramlar under bordet, sussar en minut
    Vaknar av att vinet i glaset tagit slut
\end{parse lines}

\vissteduatt{visste du att...}


\newpage

\textbf{Ölvisor}

\newpage


\subsection*{Importvisan} 
\index[alfa]{Importvisan}
\index[anfa]{T-U-B-O-R-G}
\songinfo{Mel: B.L.O.S.S.A\\
Lundakarnevalen 2022}

\begin{parse lines}[\noindent]{#1\\}
    T-U-B-O-R-G
    T-U-B-O-R-G
    T-U-B-O-R-G
    Den är importerad!
\end{parse lines}

\vissteduatt{visste du att...}


\newpage


\subsection*{Ölen är slut} 
\index[alfa]{Ölen är slut}
\index[anfa]{Ölen är slut}
\songinfo{Mel: Havet är djupt}

\colorbox{yellow}{OSÄKER PÅ DENNA}

\begin{parse lines}[\noindent]{#1\\}
    Det börjar bli sent på festen
    och alla har jävligt kul
    Du slog din rival på beerpong
    och din dans börjar bli ful

    Men du kan ej begripa
    rösten som kallar ut
    “Det där var vår sista IPA 
    och nu så är ölen slut!”
    Åh nej!

    Ölen är slut!
    Ölen är slut!
    Ända till Lomma,
    är burkarna tomma
    Ölen är slu!
    I kylskåpet finns ju ingenting,
    förutom tacos och powerking
    Så om du vill halsa,
    nöj dig med salsa
    Ölen är slut!

    Hos grannen är alla glada,
    dit floder av IPA når,
    men här på efterfesten
    rinner endast en tår

    Du mister ditt lugn i krisen
    och slutar att vara snäll
    Sen cakear du på polisen    (Byt ut cakear)
    och hamnar i fyllecell
    Åh nej!

    Ölen är slut!
    Ölen är slut!
    Så drick mer vatten,
    sov gott om natten,
    sen pustar vi ut
    I kylskåpet finns ju ingenting,
    förutom tacos och powerking
    Så om du vill halsa
    nöj dig med salsa
    Ölen är slut!
\end{parse lines}

\vissteduatt{visste du att...}


\newpage

\textbf{Snapsvisor}

\newpage


\subsection*{Feministvikingen} 
\index[alfa]{Feministvikingen}
\index[anfa]{Feministvikingen}
\songinfo{Mel: When Johnny Comes Marching Home}

\colorbox{yellow}{OSÄKER PÅ DENNA}

\begin{parse lines}[\noindent]{#1\\}
    En viking viker tvätten själv
    Hurra hurra!
    Ordet "viking" kommer sig därav, jaha!
    Föräldraledigheten delas exakt
    När vikingen sina barn har lagt
    Då är vikingens fru ute på jakt

    En viking vill ha livets vann
    Hurra hurra!
    Men på sig själv han lägger band,
    Jaha, vad bra!
    Mjödet prioriteras sist,
    Tvätta och städa blir aldrig trist
    För våran viking, han är feminist
\end{parse lines}

\vissteduatt{visste du att...}


\newpage


\subsection*{Hästens rumpa} 
\index[alfa]{Hästens rumpa}
\index[anfa]{Hästens rumpa}
\songinfo{Mel: Längtan till landet\\
Lundakarnevalen 1990}

\colorbox{yellow}{OSÄKER PÅ DENNA}

\begin{parse lines}[\noindent]{#1\\}
    Hästens rumpa skjuter stora bomber
    byxor har den inte några alls
    När jag skyfflat skit i katakomber
    känner jag mig torr uti min hals
    Häller upp en liten jävel i glaset,
    just som jag ska till att dricka den
    tänker jag på jättedjuret med aset
    Dricker tills jag glömmer honom sen
\end{parse lines}

\vissteduatt{visste du att...}


\newpage


\subsection*{Lillebror och jag} 
\index[alfa]{Lillebror och jag}
\index[anfa]{Lillebror och jag}
\songinfo{Mel: Måltidssången (refrängen)\\
Lundakarnevalen 1998}

\begin{parse lines}[\noindent]{#1\\}
    Tycker du att snapsen är för stor
    kan du ge en slatt till lillebror
    Både en, både två, både tre, både fem
    och sen blir det fosterhem!
\end{parse lines}

\vissteduatt{visste du att...}


\newpage


\subsection*{En gång i månan} 
\index[alfa]{En gång i månan}
\index[anfa]{En gång i månan}
\songinfo{Mel: Mors lilla Olle}

\begin{parse lines}[\noindent]{#1\\}
    En gång i månan är månen full,
    Men aldrig vi sett honom ramla omkull
    Stum av beundran hur mycket han tål,
    Höja vi glasen och dricka hans skål!
\end{parse lines}

\vissteduatt{visste du att...}


\newpage


\subsection*{Mjölkade en ko} 
\index[alfa]{Mjölkade en ko}
\index[anfa]{Mjölkade en ko}
\songinfo{Mel: Jag fångade en räv}

\begin{parse lines}[\noindent]{#1\\}
    Jag mjölkade en ko idag 
    Men när jag såg juvret 
    Då hade jag nog tagit fel 
    För gladast var nog tjuren
\end{parse lines}

\vissteduatt{visste du att...}


\newpage


\subsection*{Stopp en stund} 
\index[alfa]{Stopp en stund}
\index[anfa]{Stopp en stund}
\songinfo{Mel: Räven raskar över isen}

\begin{parse lines}[\noindent]{#1\\}
    Stopp en stund med skratt och pratet,
    kniv och gaffel lägg på fatet
    Seden är, att så här,
    man handskas med destilatet

    Man lyfter glaset med höger hand,
    och trycker läpparna mot dess rand
    Man dricker ur, och grinar sur,
    och väntar på resultatet
\end{parse lines}

\vissteduatt{visste du att...}


\newpage


\subsection*{Nubben görs av gran och tall} 
\index[alfa]{Nubben görs av gran och tall}
\index[anfa]{Nubben görs av gran och tall}
\songinfo{Mel: Kovan kommer}

\colorbox{yellow}{OSÄKER PÅ DENNA}

\begin{parse lines}[\noindent]{#1\\}
    Nubben görs av gran och tall,
    gran och tall, gran och tall,
    smakar bra i alla fall,
    alla fall, alla fall

    Fastän gjord av barr och grenar
    den i strupen ljuvlig lenar,
    barr och kottar, hej, gutår
    Här ska ni se hur Helan går
\end{parse lines}

\vissteduatt{visste du att Helan kan här bytas ut mot den snaps som är på tur}


\newpage


\subsection*{Mitt lilla lån} 
\index[alfa]{Mitt lilla lån}
\index[anfa]{Mitt lilla lån}
\songinfo{Mel: Hej tomtegubbar}

% \colorbox{yellow}{OSÄKER PÅ DENNA}

\begin{parse lines}[\noindent]{#1\\}
    ||: Mitt lilla lån det räcker inte
    det går till öl och till brännvin :||
    Till öl och brännvin går det åt,
    och till en pizza emellanåt
    Mitt lilla lån det räcker inte,
    det går till öl och till brännvin
\end{parse lines}

\vissteduatt{Visste du att låten från början gick “Min lilla lön” men \\
ändrades för att bättre passa studenter ekonomiska situation som \\
redan på 40-talet verkar ha varit knaper}


\newpage


\subsection*{Jag ser ni kröka(r)} 
\index[alfa]{Jag ser ni kröka(r)}
\index[anfa]{Jag ser ni kröka(r)}
\songinfo{Mel: Jag ser det snöar}

\colorbox{yellow}{OSÄKER PÅ DENNA}

\begin{parse lines}[\noindent]{#1\\}
    Jag ser ni krökar, jag ser ni krökar
    Det var ju trevligt, hurra!
    Nu blir Ni fulla, nu blir Ni fulla
    men se det blir inte jag

    För jag har huve’t på skaft
    och dricker blott saft
    För jag vill minnas den tiden
    jag med Er har haft
\end{parse lines}

\vissteduatt{visste du att...}


\newpage


\subsection*{Autocorrect} 
\index[alfa]{Autocorrect}
\index[anfa]{Vem kan skriva ett sms}
\songinfo{Mel: Vem kan segla}

\colorbox{yellow}{OSÄKER PÅ DENNA}

\begin{parse lines}[\noindent]{#1\\}
    Vem kan skriva ett sms
    efter sju-åtta snapsar
    Utan autocorrects effect
    som gör att allt kollapsar

    Jag kan skida fett sms
    efter sju-klocka snacksar
    Utan automobilskonfekt
    jag dör på katt och paxar
\end{parse lines}

\vissteduatt{visste du att...}


\newpage


\subsection*{Ångestvisan} 
\index[alfa]{Ångestvisan}
\index[anfa]{Ångestvisan}
\songinfo{Mel: Mössens julafton}

% \colorbox{yellow}{OSÄKER PÅ DENNA}

\begin{parse lines}[\noindent]{#1\\}
    Jag vaknar upp på söndan
    och bakis som ett djur
    så kommer minnen om att gårdagskvällen spårat ur
    Jag blev visst lite busig av humle och av malt,
    på snapchatstoryn har jag publicerat allt

    Fan då, jävlar, bakfyllepanik!
    För storyn min har blivit till en eländesfabrik
    Fan då, jävlar vad ska jag ta mig till?
    För allt där finns ju sparat vare sig jag vill!

    Jag hångla med min granne
    och fast det ej va rätt,
    så sålde jag hans möbler och hans katt på internet
    Jag känner mig så rutten, jag är så full av skam
    Av alla jävla bilder på min instagram!

    Fan då, jävlar…

    Jag bada i Poseidon (sjön sjön?)
    och till min stora skräck,
    så fick en barnfamilj bevittna när jag dansa näck
    Jag ramlade i Olgas, och spydde upp all gin
    och likea alla exets bilder på LinkedIn

    Fan då, jävlar…

\end{parse lines}

\vissteduatt{visste du att...Enligt utsago funnen i Lund\\
Den var omskriven till chalmers så skriv tillbaka den!}


\newpage

\textbf{Rekursiva visor}

\newpage


\subsection*{Min gode vän jäger} 
\index[alfa]{Min gode vän jäger}
\index[anfa]{Min gode vän jäger}
\songinfo{Mel: Trampa på gasen}

\colorbox{yellow}{OSÄKER PÅ DENNA}

\begin{parse lines}[\noindent]{#1\\}
    Min gode vän jäger
    Han är en god kamrat
    Han har hjorthorn fram och etikett där bak
    MIn gode vän Jäger
    Han smakar ganska gott
    Man kan ta han som en shot
    Om man vill och det vill man!
\end{parse lines}

\vissteduatt{visste du att...}


\newpage


\subsection*{Min gode vän jäger} 
\index[alfa]{Min gode vän jäger}
\index[anfa]{Min gode vän jäger}
\songinfo{Mel: Trampa på gasen}

\colorbox{yellow}{OSÄKER PÅ DENNA}

\begin{parse lines}[\noindent]{#1\\}
    Jag klarade tentan
    Jag klarade tentan
    Men bara nätt och jämnt
    Och det var tur för 
    Jag har alla pengar bränt
    På öl och spriten
    Och så på (qvinns) /klägg såklart
    Livet är helt underbart
    Så nu tar vi och super!
\end{parse lines}

\vissteduatt{visste du att...}


\newpage


\subsection*{Att tenta matstat} 
\index[alfa]{Att tenta matstat}
\index[anfa]{Att tenta matstat}
\songinfo{Mel: När månen vandrar}

\colorbox{yellow}{OSÄKER PÅ DENNA}

\begin{parse lines}[\noindent]{#1\\}
    Jag tänker tenta matstat nån gång
    jag tänker ta dom poängen
    förra tentan blev inte lång
    jag borde stannat i sängen
    Den jävla kursen, den gör mig galen
    Jag tror jag skyller på karnevalen
    och tar den sen i framtiden!
\end{parse lines}

\vissteduatt{visste du att...}


\newpage


\subsection*{Groggen} 
\index[alfa]{Groggen}
\index[anfa]{Groggen}
\songinfo{Peter "Pun man" Seimar (F)\\
Kategori: SåS Refuserade 2011}

% \colorbox{yellow}{OSÄKER PÅ DENNA}

\begin{parse lines}[\noindent]{#1\\}
    Jag har nu blandat en helt ny grogg,
    som gör mig full som en tunna.
    Receptet skriver jag på min blogg,
    så att andra ska kunna
    En Rom-o-cola, fast utan cola
    och mera rom, för vi ska ej snåla
    En god idé, den blir succé 
\end{parse lines}

\vissteduatt{visste du att...}


\newpage


\subsection*{Jag drikker Tuborg} 
\index[alfa]{Jag drikker Tuborg}
\index[anfa]{Jag drikker Tuborg}
\songinfo{Mel: Trampa på gasen\\
synge på dansk}

\begin{parse lines}[\noindent]{#1\\}
    Jeg drikker Tuborg
    og snapser gammeldansk
    Det ær alt jeg vil,
    det ær alt jeg kan
    Jeg drikker Tuborg
    og snapser gammeldansk
    Det ær alt jeg vil og kan,
    så hold kæft jeg ær lykkelig!
\end{parse lines}

\vissteduatt{visste du att...}


\newpage

\textbf{Internationella visor}

\newpage


\subsection*{This is the wine} 
\index[alfa]{This is the wine}
\index[anfa]{This is the wine}
\songinfo{Mel: This is the way\\
Lundakarnevalen 2014}

\begin{parse lines}[\noindent]{#1\\}
    This
    This is the wine
    This is the wine we’ll whine about
    When we wake up tomorrow (wakin’ up, wakin’ up)
    This is the wine we’ll whine about!
\end{parse lines}

\vissteduatt{visste du att...}


\newpage


\subsection*{Follow me} 
\index[alfa]{Follow me}
\index[anfa]{This looks like a shot for me}
\songinfo{Mel: Eminem - Without me (refräng)\\
Lundakarnevalen 2022}

\begin{parse lines}[\noindent]{#1\\}
    Now this looks like a shot for me
    So everybody, just follow me
\end{parse lines}

\vissteduatt{visste du att...}


\newpage


\subsection*{La' oss drikke} 
\index[alfa]{La' oss drikke}
\index[anfa]{La' oss drikke}
\songinfo{Mel: Punschen kommer}

\colorbox{yellow}{OSÄKER PÅ DENNA}

\begin{parse lines}[\noindent]{#1\\}
    Gamle venner halsen brenner
    Øl – øl – øl
    Hvilken væske kan oss leske?
    Øl – øl – øl
    Er vår sang en ikke ren og klas som sølv,
    la oss drikke, la oss drikke
    Øl – øl – øl
\end{parse lines}

\vissteduatt{visste du att...}


\newpage

\textbf{Sektionsvisor}

\newpage


\subsection*{Av med ouverallen} 
\index[alfa]{Av med ouverallen}
\index[anfa]{Av med ouverallen}
\songinfo{Melodi: Ingen har sett fantomen\\
Från A-nollningen 2010}

% \colorbox{yellow}{OSÄKER PÅ DENNA}

\begin{parse lines}[\noindent]{#1\\}
    Ingen har sett ett I-Phøs utan pengar,
    kuddar av sedlar, kashmir i sina sängar
    Han drar pappas kort på Lunds nation
    när han vaskar champagnen i hon

    Ingen har sett ett W-Phøs gå i otakt,
    svin mot sina nollor, gråter över valjakt
    Dreadlocks, tofu \& minkar ut ur bur,
    blir adopterade av Moder Natur

    Av med ouverallen, det är inte minusgrader,
    du är över tio, jobbar inte på ett lager
    Gör som A-Phøs skaffa en kavaj, 
    det ser bättre ut på varje partaj!

    Ingen har sett ett E-Phøs hela och rena,
    bajs i håret och lera över bena
    Hygien gör väl ingenting, 
    det finns ändå inga damer omkring

    Ingen har sett ett D-Phøs mitt på dagen,
    är framför Warcraft, ständigt helt upptagen
    IRL är ett skrämmande begrepp,
    men kvinnorna på Sims gör mig pepp

    Av med ouverallen…

    Ingen har sett ett K-Phøs stå och skratta,
    skriker och gormar, hur fan ska nollan fatta?
    Kärleksbrygd – enda chans för en druid,
    till att bryta LTH:s nollefrid

    Ingen har sett ett ING-Phøs
    …
    …
    …

    Av med ouverallen…

    Ingen vill se ett M-Phøs sjukjournalerm
    AIDS, streptokocker och täppta tarmkanaler
    Försöker penetrera så fort han bara kan,
    en parodi på en tvättäkta man

    En mamma och en pappa fick ett klychigt foster, 
    villa, Volvo, vovve, till gudmor fick han moster
    IQ under medel, kär i en blond tös,
    denna pojke är numera V-Phøs

    Av med ouverallen…

    Ingen har sett ett F-Phøs innan teknis, 
    högstadiet var en enda social miss
    Alkohol, bröst och en och annan dans,
    Skåne är eran sista chans

    Aldrig har ett A-Phøs skämtat om sig själva,
    har inte tid, är i skolan minst till elva
    Klipper och klistrar får ändå CSN,
    och linjalerna får ersätta män

    Vi vill ju faktiskt också ha en sparkdräkt,
    önskar gemenskap i eran LTH-sekt
    Ber om förlåtelse genom alkohol, 
    och utbringar för er en stor skål!
\end{parse lines}

\vissteduatt{Visste du att A-sektionen skaffade ouveraller först 20XX, innan dess användes endast kavaj.\\
Visste du att BME-programmet kom till först 2011, alltså efter denna låt skrevs}


\newpage


\subsection*{Rosa på bal} 
\index[alfa]{Rosa på bal}
\index[anfa]{Rosa på bal}
\songinfo{Text: Evert Taube\\
D-sektionens sektions hymn}

% \colorbox{yellow}{OSÄKER PÅ DENNA}

\begin{parse lines}[\noindent]{#1\\}
    Tänk att jag dansar med Andersson,
    lilla jag, lilla jag,
    med Fritiof Andersson!
    Tänk att bli uppbjuden av en så'n
    populär person!

    Tänk, vilket underbart liv, det Ni för!
    Säg mig, hur känns det att vara charmör,
    sjöman och cowboy, musiker, artist...
    Det kan väl aldrig bli trist?

    Nej, aldrig trist, fröken Rosa,
    har man som Er kavaljer
    Vart jag än ställer min kosa,
    aldrig förglömmer jag Er

    Ni är en sångmö från Helikons berg
    Åh, fröken Rosa, er linje, er färg,
    skuldran, profilen med lockarnas krans,
    ögonens varma glans!

    Tänk, inspirera herr Andersson,
    lilla jag, inspirera Fritiof Andersson!
    Får jag kanhända min egen sång,
    lilla jag, nå'n gång?

    "Rosa på bal", vackert namn, eller hur?
    Början i moll och finalen i dur
    När blir den färdig, herr Andersson, säg,
    visan Ni diktar till mig?

    Visan om Er, fröken Rosa,
    får Ni ikväll till Ert bord
    Medan vi talar på prosa
    diktar jag rimmande ord

    Tyst! Ingen såg att jag kysste Er kind
    Känn hur det doftar från parken av lind!
    Blommande lindar kring månbelyst stig...
    Rosa, jag älskar dig!
\end{parse lines}

\vissteduatt{Visste du att alla Rosa sjunges \\som färgen råsa. Rimmen modifieras därefter.}


\newpage

\textbf{Slaskvisor}

\newpage


\subsection*{En galen natt} 
\index[alfa]{En galen natt}
\index[anfa]{En galen natt}
\songinfo{Mel: En kulen natt\\
I-sektionen Sångarstriden 2009}

% \colorbox{yellow}{OSÄKER PÅ DENNA}

\begin{parse lines}[\noindent]{#1\\}
    En galen natt, natt, natt, 
    jag shotta fläder, 
    och det blev shot efter shot efter shot, 
    sen blev det jäger
    Och vart jag sågade, sågade, såg 
    så såg jag två gånger två gånger två, 
    och under bordet i dimman jag föll 
    där låg en till och det var du! 

    Vi tagga till som få,
    och intog natten, 
    och det blev dans efter dans efter dans, 
    men det blev svart sen
    Vi vakna upp mitt på dan, mitt på dan 
    i en rabatt i Botan, i Botan 
    och till vår glädje bland blomster och skit, 
    låg en flaska sprit och vi sa SKÅL!
\end{parse lines}

\vissteduatt{Visste du att...}


\newpage


\subsection*{Jag var full en gång} 
\index[alfa]{Jag var full en gång}
\index[anfa]{Jag var full en gång}
\songinfo{Mel: Flottarkärlet}

\colorbox{yellow}{OSÄKER PÅ DENNA}

\begin{parse lines}[\noindent]{#1\\}
    Jag var full en gång för länge sen,
    på knäna kröp jag hem
    Och på vägen träffa jag en elefant, elefant
    Elefanten börja pinka och jag trodde det var öl,
    sedan dess har jag bli'tt kalald jävla knöl, mera öl!




    (kanske bara första versen?)




    Jag var full en gång för länge sen,
    på knäna kröp jag hem
    varje dike var för mig ett vilohem, vilohem
    I varje hörn och varje vrå så hade jag en liten vän,
    ifrån renat upp till 96 procent, mera sprit!
\end{parse lines}

\vissteduatt{Visste du att...}


\newpage



