\begin{center}
    \vspace*{1.5cm}
    {\fontsize{20}{20}\textbf{Skånska visor}}\\
    \vspace{0.7cm}
    {\fontsize{12}{12}\textit{Om Edvard Persson/skåningen/lundabon/lunditen själv får välja}}
\end{center}
\addtocwithheader{Skånska visor}  % Add entry to TOC and set header\noBackground
\noBackground

\newpage
\resetBackground


\subsection*{Vi klarar oss nog ändå} 
\index[alfa]{Vi klarar oss nog ändå}
\index[anfa]{Vi klarar oss nog ändå}
\songinfo{Text: Lasse Dahlquist}

\begin{parse lines}[\noindent]{#1\\}
    Jag vill sjunga en visa i klaraste dur,
    ty den handlar om Skåne och slätter och djur
    Kan hända den retar en del 
    men i så fall är det deras eget fel
    Det har talats så mycket om dynga och lort 
    men betänk vilken oerhörd nytta den gjort
    ||: Så låt dom bara gå på vi klarar oss nog ändå :||
    % Ja låt dom bara gå på vi klarar oss nog ändå

    Kanske språket vi talar ej klingar så väl,
    men det är och förbliver en del av vår själ
    Kan hända det retar en del,
    men i så fall är det deras eget fel
    Uti självaste riksda'n på skånska dom slåss,
    för de flesta utav dom har kommit från oss
    ||: Så låt dom... :||
    % Så låt dom bara gå på...

    Utav våra produkter de smörjer sitt krås,
    och de är ifrån oss som dom fått Mårten Gås
    Kan hända det retar en del,
    men i så fall är det deras eget fel
    Det har klagats på bostaden på våra svin
    men när julskinkan kommer jo då är den fin
    ||: Så låt dom... :||
    % Så låt dom bara gå på...
\end{parse lines}

\begin{parse lines}[\noindent]{#1\\}
    Hela landet får njuta av vår akvavit
    Sockerbetan har lärt dom att dricka på bit
    Kan hända det retar en del,
    men i så fall är det deras eget fel
    Våran sandstrand den är både bländvit och fin
    och så har vi ju vår lilla vita kanin
    ||: Så låt dom... :||
    % Så låt dom bara gå på...

    Selma Lagerlöv som är en fin gammal dam,
    med Nils Holgersson gjorde för Skåne reklam
    Kan hända det retar en del,
    men i så fall är det deras eget fel
    % Kan hända det retar…
    Tänk sån nytta som storken från Skåne har gjort,
    men det hindrar ju inte att folk pratar lort
    Men låt dem bara gå på vi klarar oss nog ändå
\end{parse lines}

\vissteduatt{Visste du att sjunges på skånska?...}

\newpage


\subsection*{Skåne} 
\index[alfa]{Skåne}
\index[anfa]{Nu har det blivit dags att dricka skåne}
\songinfo{Mel: Lite grann från ovan}

\begin{parse lines}[\noindent]{#1\\}
    Nu har det blivit dags att dricka Skåne
    En liten sträv, så gyllengul som raps
    Det är det bästa mellan sol och måne
    Som lätt får mången stark till snabb kollaps
    Så fatta nu din hand om hela Skåne
    Här kommer södra Sverige i en snaps
    Känn hur Lund och Smygehuk
    slår små volter i din buk
    Aquaviten ska va' gul
    och heta Skåne
\end{parse lines}

\vissteduatt{Visste du att fler skånska snapsvisor finns på sida…}


\newpage


\subsection*{Till den skånska metropolen Vinslöv} 
\index[alfa]{Till den skånska metropolen Vinslöv}
\index[anfa]{Vinslöv}
\songinfo{Mel: Wiensk operettvals}

\colorbox{yellow}{OSÄKER PÅ DENNA}

\begin{parse lines}[\noindent]{#1\\}
    I vin, i vin, i vin, i vin,
    i Vinslöv bor min mor
    På Hven, på Hven, på Hven, på Hven,
    påven han bor i Rom
    I Rom, i Rom, i Rom, i Rom,
    i rompan på en ko
    I ko, i ko, i ko, i ko,
    i Kosta göra man glas
    I glas, i glas, i glas, i glas,
    i glas där har man vin
    I vin, i vin, i vin, i vin,
    i Vinslöv bor min mor
\end{parse lines}

\vissteduatt{visste du att...}

\newpage



\subsection*{Landskrona} 
\index[alfa]{Landskrona}
\index[anfa]{I Landskrona}
\songinfo{Mel: In the Ghetto - Elvis Presley}

\begin{parse lines}[\noindent]{#1\\}
    På en regnig strand
    Längs Skånes gråa västra kust 
    Bor några tusen fast dom inte har lust
    I Landskrona
    (I Landskrona)
    
    Flytta gärna dit
    Våran utsikt den är ganska käck
    Om man bortser ifrån Barsebäck
    Och det gör man
    (Inte riktigt)
    
    På krogen varje fredagskväll
    Där hänger Per och Kjäll
    Två fruktade lynniga bröder ifrån Ven
    
    Det är en kille ifrån Höör som stör
    Han har snott deras rullebör
    Nu ska dom ge han vad han tål med rör av rostfritt stål
    
    Sicket tuftt klimat
    Men det finns en sak som förenar oss 
    Det är stadens stolthet BoIS förstås
    Heja svartvitt
    (Heja svartvitt)
    
    Att spela boll är kul
    På Landskrona IP står ett randigt gäng
    Dom har än en gång fått rejält med däng
    I Landskrona
    (Utav Mjellby)
    
    Sånna dagar känns det skit att va landskronit
    Vill ta pågatåget bort från stan
    Men på just den dan kan man ge sig fan
    På att det är strejk
    
    Så man vandrar hem en kulen kväll
    Och man träffar Kjell som åkt på en smäll
    Så han blör
    (Jäkla Höör)
    
    Men de ger aldrig upp
    För nästa helg tar vi nya tag
    Både BoIS och Per och Kjell och jag
    I Landskrona
    
\end{parse lines}

\vissteduatt{visste du att...}

\newpage



\subsection*{Lite grann från ovan} 
\index[alfa]{Lite grann från ovan}
\index[anfa]{Lite grann från ovan}
\songinfo{Text: Lasse Dahlqvist\\
Sång: Edvard Persson}

% \colorbox{yellow}{OSÄKER PÅ DENNA}

\begin{parse lines}[\noindent]{#1\\}
    Jag är en liten gåsapåg från Skåne
    En skåning som Ni vet är alltid trygg
    Och fast jag är så nära sol och måne
    Jag sitter säkert på min gåsarygg
    Långt under mig det ligger som en tavla
    Det vackraste i världen man kan se
    Både skogar, sjö och strand
    Blir ett enda sagoland
    När man ser det lite grann så här från ovan

    Där ligger gamla slott och härresäten
    Som minnen från den stolta tid som flytt
    Och aldrig skall den tiden bli förgäten
    Men inget slag skall stånda här på nytt
    Nej, dessa fält skall bära samma skördar
    Som de har gjort i sekelflydda dar
    Ja det är min liv och kniv
    Alla tiders perspektiv
    När man ser det lite grann så här från ovan

    Du kära gås som stolt i skyn dig svingar
    Har ingen farlig last att kasta ned
    Ty du bär fredens vita vackra vingar
    Som världen längtar efter mer och mer
    När människobarnen går därnere och kivas
    Då resonerar du nog liksom jag
    Tänk vad skönt det är ändå
    Att få sväva i det blå
    Och se Er lite grann så här från ovan

    Nu jordens alla murar börjar skaka
    En samling dårar satt vår värld i brann
    Vad fäderna byggt upp blir pannekaka
    Det ryck och slits i gamla vänskapsband
    Ack kära Ni som slåss därner på jorden
    Kom upp och ta en liten titt med mig
    Jag är ganska säker på
    Att Ni skäms en smula då
    När Ni ser vår gamla jord så här från ovan
\end{parse lines}

\vissteduatt{visste du att...}


\newpage 


\subsection*{Eslövs nationalsång} 
\index[alfa]{Eslövs nationalsång}
\index[anfa]{Eslövs nationalsång}
\songinfo{Jag kan inte melodin}

\begin{parse lines}[\noindent]{#1\\}
    ||: Vi går här på slätten
    och vi hackar våra bedor.
    Vi går här på slätten och hackar hela dan :||
    ||: Åååh, hackar våra bedor
    Åååh, hackar hela dan :||

    ||: Jag ska vända mig, och bocka mig,
    och ta en liten beda.
    Jag ska vända mig, och bocka mig,
    och ta en beda till :||
    ||: Åååh, jag tar en liten beda
    Åååh, jag tar en beda till :||
\end{parse lines}

\colorbox{blue}{Hur går melodin?}

\vissteduatt{Visste du att...}

\newpage