\begin{center}
    \vspace*{1.5cm}
    {\fontsize{20}{20}\textbf{Andra sektioners visor (KAMPVISOR?)}}\\
    \vspace{0.7cm}
    {\fontsize{12}{12}\textit{Om [insert sektion] själv får välja}}
  
  \end{center}
  \noBackground
  
  \newpage
  \resetBackground

\subsubsection*{Kul info om andra sektioner}

Här ska vi skriva kul info om andra sektioner och att de har maskotar mm.

Vill vi skriva om TLTH också och att de också har färg och djur?

Uppmana till att andra visor är bra. cos(x) bra för endimen.

\newpage


% Detta fungerar inte riktigt än. Något blir knas med krusidull-E:et på varannan sida
\backgroundsetup{
  scale=0.40,
  opacity=0.2,
  angle=0,
  color=black,
  vshift=-230,
  hshift=160,
  contents={\includegraphics[width=\paperwidth]{./bilder/large_F.png}}
}

\subsection*{Hyllningsvisa till teknisk fysik} 
\index[alfa]{Hyllningsvisa till teknisk fysik}
\index[anfa]{Teknisk Fysik är mössbeklädda töntar}
\songinfo{Mel: Sit on my face}

\begin{parse lines}[\noindent]{#1\\}
    Teknisk Fysik är mössbeklädda töntar,
    fula flickor och en samling mammas pojkar.
    Liknar mest en televerksbil, som gått för många mil,
    en teknisk fossil.

    Teknisk Fysik är lättare än att fjärta,
    döda älgar värmer nu mitt kalla hjärta.
    Ta hit dynamit, spräng teknik, vårt gebit.
    Dessa tofsprydda avskum som ger oss kolik!
    Dessa tofsprydda avskum som ger oss kolik!
    Dessa tofsprydda avskum som ger oss kolik!
    Dessa tekniska lik!
    Barambam!


\end{parse lines}

\vissteduatt{Visste du att de första E:arna nollades av F:are?}


\newpage
\resetBackground
% \noBackground


\subsection*{Vi går på F-sek} 
\index[alfa]{Vi går på F-sek}
\index[anfa]{kosex}
% \songinfo{Mel: Sit on my face}

\begin{parse lines}[\noindent]{#1\\}
    Vi går på F-sek, F-sek vi går på F-sek
    Derivera Sin(x) så får du Cos(x)
    Cos(x) Cos(x) vi vill ha Cos(x)
    Vill vi? Neeej!
    Vi går på F-sek, F-sek vi går på F-sek
    Derivera Sin(x) så får du Cos(x)
    Cos(x) Cos(x) vi vill ha Cos(x)
    Vill vi? Neeej, vi vill ha ÄLGSEX! \textbf{också?}

\end{parse lines}

\vissteduatt{Denna är bra att komma ihåg till mattetentor}


\newpage


\subsection*{Supa tills vi stupar} 
\index[alfa]{Supa tills vi stupar}
\index[anfa]{---}
\songinfo{Mel: The wild rover}

\begin{parse lines}[\noindent]{#1\\}
    ...

\end{parse lines}


\subsection*{Vår färg är röd} 
\index[alfa]{Vår färg är röd}
\index[anfa]{Vår färg är röd}
\songinfo{Mel: When the saints go marching in}

\begin{parse lines}[\noindent]{#1\\}
    Vår färg är röd, vår färg är fin,
    för det är vi som går Maskin
    Och vi har kommit för att dricka alkohol,
    för det är vi som går Maskin

\end{parse lines}

\vissteduatt{fint om M-sek}


\newpage
% \resetBackground % ------------------------------------------------------- reset background


\subsection*{V-ingenjören skapelsens krona} 
\index[alfa]{V-ingenjören skapelsens krona}
\index[anfa]{V-ingenjören skapelsens krona}
\songinfo{Mel: Sovjets nationalsång (nästan)}

\begin{parse lines}[\noindent]{#1\\}
    Gudarnas gunstling, så stark, klok och sann.
    Från falsk blygsamhet vi ska er förskona, 
    vi vet vad vi vill och vi vet vad vi kan. 

    Vi bygger bro mellan fjärran stränder. 
    Vi drar en väg mellan alla länder.
    En väg mellan folken, en strålande syn.
    Vi skapar kraft, tämjer syndafloden. 
    Vår jord bebyggt så man kan bebo den, 
    Från klingfasta berget mot strålande blånande? skyn.

    Skåla kamrater, skål för varandra. 
    Skål ingenjörer av sten och av stål. 
    Vi breddar vartefter den vägen vi vandra. 
    Vårt mål är en skål, så skål för vårt mål. 

    Vi bygger bro…

\end{parse lines}

\vissteduatt{fint om V-sek}


\newpage


\subsection*{Brand, lant, väg och vatten} 
\index[alfa]{Brand, lant, väg och vatten}
\index[anfa]{Brand, lant, väg och vatten}
% \songinfo{Mel: Sovjets nationalsång (nästan)}

\begin{parse lines}[\noindent]{#1\\}
    Brand, lant, väg och vatten,
    Störst på LTH
    Ingenting kan stoppa oss
    V-sek alé alé!

\end{parse lines}

% \vissteduatt{fint om V-sek}

\subsection*{För vi är de blå} 
\index[alfa]{För vi är de blå}
\index[anfa]{För vi är de blå}
\songinfo{Melodi: Das rote Pferd}

\begin{parse lines}[\noindent]{#1\\}
    ||: FÖR… VI… ÄR… de blå, och vi är inte små
    Nej vi störst och bäst på hela LTH
    Shalalala lala
    Shalalala lala
    Shalalala la la la la lalalalala :||

\end{parse lines}


\newpage


\subsection*{A-sek} 
\index[alfa]{A-sek}
\index[anfa]{A-sek}
% \songinfo{Mel: Sovjets nationalsång (nästan)}

\begin{parse lines}[\noindent]{#1\\}
    A-sek

\end{parse lines}

\vissteduatt{fint om A-sek}

\newpage


\subsection*{K-sek} 
\index[alfa]{K-sek}
\index[anfa]{K-sek}
% \songinfo{Mel: Sovjets nationalsång (nästan)}

\begin{parse lines}[\noindent]{#1\\}
    K-sek

\end{parse lines}

\vissteduatt{fint om K-sek}

\newpage


\subsection*{D-sek} 
\index[alfa]{D-sek}
\index[anfa]{D-sek}
% \songinfo{Mel: Sovjets nationalsång (nästan)}

\begin{parse lines}[\noindent]{#1\\}
    D-sek

\end{parse lines}

\vissteduatt{fint om D-sek}


\newpage


\subsection*{Vi är ING-sektionen} 
\index[alfa]{Vi är ING-sektionen}
\index[anfa]{Vi är ING-sektionen}
\songinfo{Mel: We will rock you}

\begin{parse lines}[\noindent]{#1\\}
    ...

\end{parse lines}


\subsection*{ING från sundets pärla} 
\index[alfa]{ING från sundets pärla}
\index[anfa]{För vi är ING från sundets pärla}
\songinfo{Mel: Robin Hood Rooster song}

\begin{parse lines}[\noindent]{#1\\}
    För vi är ING från sundets pärla
    Och det är fest idag igen
    Och vi ska supa hela natten lång,
    och sjunga denna sång
    Shalalala…


\end{parse lines}

\vissteduatt{fint om ING-sek}


\newpage


\subsection*{Eko halleluja (Kolla med W vad som gäller nu)} 
\index[alfa]{W-sek}
\index[anfa]{W-sek}
\songinfo{Mel: Glory Halleluja}

\begin{parse lines}[\noindent]{#1\\}
    W-sek

\end{parse lines}

\subsection*{Ekosång???} 
\index[alfa]{Ekosång}
\index[anfa]{Ekosång}
% \songinfo{Mel: Glory Halleluja}

\begin{parse lines}[\noindent]{#1\\}
    Aha, ekosång! Alla eko:sar kom igång! (x5)

\end{parse lines}

\vissteduatt{fint om W-sek}


\newpage


\subsection*{Turkosa samban} 
\index[alfa]{Turkosa samban}
\index[anfa]{Turkosa samban}
\songinfo{Mel: Samba de Janeiro\\
(Sungs förslagsvis tsm med playback)??????????
}

\begin{parse lines}[\noindent]{#1\\}
    ||: Turkos Turkos här gör vi entré
    Vi bränner sprit uti KC:G
    Turkos Turkos så gör dubbel W
    Mitt glas är två för att jag är sne! :||
    ||: Simma sí, ah simma, ah simma simma simma :|| x4

\end{parse lines}


\subsection*{Hej här kommer W-sektionen} 
\index[alfa]{Hej här kommer W-sektionen}
\index[anfa]{Hej här kommer W-sektionen}
\songinfo{Mel: Segertåget}

\begin{parse lines}[\noindent]{#1\\}
    Hej här kommer W-sektionen
    sitter allra högst på tronen
    Pantar burkar, räddar världen
    och har jävligt kul på färden
    Äh vi kör igen!

\end{parse lines}

\vissteduatt{fint om W-sek}


\newpage


\subsection*{Bortom vägar och vatten (Fråga I vilka som gäller)} 
\index[alfa]{Bortom vägar och vatten (Fråga I vilka som gäller)}
\index[anfa]{Bortom vägar och vatten (Fråga I vilka som gäller)}
\songinfo{Mel: Pomp and Circumstance March No. 1}

\begin{parse lines}[\noindent]{#1\\}
    Bortom vägar och vatten, långt långt över Maskin,
    innan Brand hunnit släckas, blickar vi ner på Kemi

    Bam bam bam!

    Stolta står vi och väntar, tills Elektro gjort sin sorti,
    för vi är bästa sektionen, och vi älskar vårat I!


\end{parse lines}


\subsection*{I-OVERALL PÅ} 
\index[alfa]{I-OVERALL PÅ}
\index[anfa]{I-OVERALL PÅ}
\songinfo{Mel: Millenium två
Text: Uppdragsgruppen Cheerleading 2011.}

\begin{parse lines}[\noindent]{#1\\}
    Åh, I overall på,
    Elitnivå,
    Top of the line på LTH
    Vi spårar som få,
    När I-Tåget köttar på!

\end{parse lines}

\vissteduatt{fint om I-sek}


\newpage