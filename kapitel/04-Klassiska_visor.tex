\begin{center}
    \vspace*{1.5cm}
    {\fontsize{20}{20}\textbf{Klassiska visor}}\\
    \vspace{0.7cm}
    {\fontsize{12}{12}\textit{Om Taube själv får välja}}
  
    \end{center}
    \noBackground

    \newpage
    \resetBackground


\subsection*{Måltidssången} 
\index[alfa]{Måltidssången}
\index[anfa]{Så lunka vi så småning om}
\songinfo{C. M. Bellman Fredmans epistel nr. 21???????}

\begin{parse lines}[\noindent]{#1\\}
    Så lunka vi så småningom,
    från Bacchi buller och tumult
    när döden ropar: Granne, kom,
    ditt timglas är nu fullt
    Du gubbe fäll din krycka ner
    och du, du yngling lyd min lag:
    den skönsta nymf som åt dig ler
    inunder armen tag

    Tycker du att graven är för djup,
    nå välan, så tag dig då en sup,
    tag dig sen dito en, dito två, dito tre
    så dör du nöjdare

    Säg, är du nöjd, min granne säg?
    Så prisa värden nu till slut
    om vi har en och samma väg,
    så följoms åt... Drick ut!
    Men först med vinet rött och vitt,
    för vår värdinna bugom oss
    och halkom sen i graven fritt
    vid aftonstjärnans bloss

    Tycker du…
\end{parse lines}

\vissteduatt{Visste du att sjunges vid varmrätt? kroka i arm?\\}

% \newpage

\subsection*{Den blomstertid nu kommer} 
\index[alfa]{Den blomstertid nu kommer}
\index[anfa]{Den blomstertid nu kommer}
% \songinfo{C. M. Bellman Fredmans epistel nr. 21???????}

\begin{parse lines}[\noindent]{#1\\}
    Den blomstertid nu kommer
    med lust och fägring stor
    Nu nalkas ljuvlig sommar
    då gräs och gröda gror
    Den blida solen väcker
    allt det som varit dött
    Den allt med grönska täcker,
    och allt blir återfött

    De fagra blomsterängar
    och åkerns ädla säd,
    de rika örtesängar
    och alla gröna träd
    skall oss var dag påminna
    Guds godhets rikedom
    Låt oss den nåd besinna
    som räcker året om

\end{parse lines}

% \vissteduatt{Visste du att sjunges vid varmrätt? kroka i arm?}


\newpage


\subsection*{Studentsången} 
\index[alfa]{Studentsången}
\index[anfa]{Sjung om studenten}
\songinfo{Musik: Prins Gustav\\
Text: Herman Sätherberg
}

\begin{parse lines}[\noindent]{#1\\}
    Sjung om studentens lyckliga dag,
    låtom oss fröjdas i ungdomens vår!
    Än klappar hjärtat med friska slag,
    och den ljusnande framtid är vår.
    ||:Inga stormar än
    i våra sinnen bo,
    hoppet är vår vän,
    och vi dess löften tro,
    när vi knyta förbund i den lund,
    där de härliga lagrarna gro!
    där de härliga lagrarna gro!:||
    Hurra!

\end{parse lines}

% \vissteduatt{Visste du att sjunges vid varmrätt? kroka i arm?}


\newpage


\subsection*{O, gamla klang och jubeltid} 
\index[alfa]{O, gamla klang och jubeltid}
\index[anfa]{O, gamla klang och jubeltid}
\songinfo{Mel: O, alte Burschenherrlichkeit!
}

\begin{parse lines}[\noindent]{#1\\}
    O, gamla klang och jubeltid, 
    ditt minne skall förbliva
    och än åt livets bistra strid 
    ett rostigt skimmer giva!
    Snart tystnar allt vårt yra skämt,
    vår sång blir stum, vårt glam förstämt;
    o jerum, jerum, jerum,
    o, quae mutatio rerum!

    Var äro de, som kunde allt,
    blott ej sin ära svika,
    som voro män av äkta halt
    och världens herrar lika?
    De drogo bort från vin och sång
    till vardagslivets tråk och tvång;
    o, jerum, jerum, jerum,
    o, quae mutatio rerum!

\end{parse lines}

\begin{parse lines}[\noindent]{#1\\}
    En tämjer forsens vilda fall,
    en annan ger oss papper,
    en idkar maskinistens kall,
    en mästrar volt så tapper,
    en ritar hus, en mäter mark,
    en blandar hop mixtur så stark;
    o, jerum, jerum, jerum,
    o, quae mutatio rerum!

    Men hjärtat i en sann student
    kan ingen tid förfrysa,
    den glädjeeld, som där han har tänt,
    hans hela liv skall lysa.
    Det gamla skalet brustit har,
    men kärnan finnes frisk dock kvar,
    och vad han än må mista,
    den skall dock aldrig brista!

    Så sluten, bröder, fast vår krets
    till glädjens värn och ära!
    Trots allt vi tryggt och väl tillfreds
    vår vänskap trohet svära.
    Lyft bägarn högt, och klinga, vän!
    De gamla gudar leva än
    bland skålar och pokaler,
    bland skålar och pokaler!


\end{parse lines}

\vissteduatt{Visste du att den tredje versen är LTH:s och KTH:s egna vers?}


\newpage


\subsection*{skippar: Dansen den går uppå Svinnsta skär} 
% \index[alfa]{O, gamla klang och jubeltid}
% \index[anfa]{O, gamla klang och jubeltid}
% \songinfo{Mel: O, alte Burschenherrlichkeit!
% }

\begin{parse lines}[\noindent]{#1\\}
    tar bort

\end{parse lines}

\subsection*{skippar: I natt jag drömde (originalet)} 
% \index[alfa]{O, gamla klang och jubeltid}
% \index[anfa]{O, gamla klang och jubeltid}
% \songinfo{Mel: O, alte Burschenherrlichkeit!
% }

\begin{parse lines}[\noindent]{#1\\}
    tar bort

\end{parse lines}

\newpage


\subsection*{Längtan till landet} 
\index[alfa]{Längtan till landet}
\index[anfa]{Vintern rasat ut bland våra fjällar}
\songinfo{Musik: O. Lindblad\\ Text: H. Sätherberg}

\begin{parse lines}[\noindent]{#1\\}
    Vintern rasat ut bland våra fjällar,
    drivans blommor smälta ned och dö
    Himlen ler i vårens ljusa kvällar
    solen kysser liv i skog och sjö
    //: Snart är sommar´n här! I purpurvågor,
    guldbelagda, azurskiftande
    ligga ägnarne i dagens lågor
    och i lunden dansa källorne ://

    Ja, jag kommer! Hälsen glada vindar
    ut till landet, ut till fåglarne,
    att jag älskar dem, till björk och lindar,
    sjö och berg, jag vill dem återse
    //: Se dem än som i min bardoms stunder
    följa bäckens dans till klarnad sjö
    Trastens sång i furuskogens lunder,
    vattenfågelns lek kring fjärd och ö ://
\end{parse lines}

\newpage


\subsection*{skippar: Nu grönskar det} 
% \index[alfa]{O, gamla klang och jubeltid}
% \index[anfa]{O, gamla klang och jubeltid}
% \songinfo{Mel: O, alte Burschenherrlichkeit!
% }

\begin{parse lines}[\noindent]{#1\\}
    tar bort

\end{parse lines}

\subsection*{skippar: Än en gång däran} 
% \index[alfa]{O, gamla klang och jubeltid}
% \index[anfa]{O, gamla klang och jubeltid}
% \songinfo{Mel: O, alte Burschenherrlichkeit!
% }

\begin{parse lines}[\noindent]{#1\\}
    tar bort

\end{parse lines}

\newpage


\subsection*{Flyttar: Vi klarar oss nog ändå} 
% \index[alfa]{O, gamla klang och jubeltid}
% \index[anfa]{O, gamla klang och jubeltid}
% \songinfo{Mel: O, alte Burschenherrlichkeit!
% }

\begin{parse lines}[\noindent]{#1\\}
    flyttar till nytt kapitel 'skånska visor'

\end{parse lines}

\newpage

