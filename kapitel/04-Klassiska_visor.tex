\begin{center}
    \vspace*{1.5cm}
    {\fontsize{20}{20}\textbf{Klassiska visor}}\\
    \vspace{0.7cm}
    {\fontsize{12}{12}\textit{Om Taube själv får välja}}
  
\end{center}
\noBackground

\newpage
\resetBackground



\subsection*{Måltidssången} 
\index[alfa]{Måltidssången}
\index[anfa]{Så lunka vi så småning om}
\songinfo{C. M. Bellman Fredmans epistel nr. 21???????}

\begin{parse lines}[\noindent]{#1\\}
    Så lunka vi så småningom,
    från Bacchi buller och tumult
    när döden ropar: Granne, kom,
    ditt timglas är nu fullt
    Du gubbe fäll din krycka ner
    och du, du yngling lyd min lag:
    den skönsta nymf som åt dig ler
    inunder armen tag

    Tycker du att graven är för djup,
    nå välan, så tag dig då en sup,
    tag dig sen dito en, dito två, dito tre
    så dör du nöjdare

    Säg, är du nöjd, min granne säg?
    Så prisa värden nu till slut
    om vi har en och samma väg,
    så följoms åt... Drick ut!
    Men först med vinet rött och vitt,
    för vår värdinna bugom oss
    och halkom sen i graven fritt
    vid aftonstjärnans bloss

    Tycker du…

\end{parse lines}

\vissteduatt{Visste du att sjunges vid varmrätt? kroka i arm?}

