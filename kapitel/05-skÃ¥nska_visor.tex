\begin{center}
    \vspace*{1.5cm}
    {\fontsize{20}{20}\textbf{Skånska visor}}\\
    \vspace{0.7cm}
    {\fontsize{12}{12}\textit{Om Edvard Persson/skåningen/lundabon/lunditen själv får välja}}
  
    \end{center}
    \noBackground

    \newpage
    \resetBackground


\subsection*{Vi klarar oss nog ändå} 
\index[alfa]{Vi klarar oss nog ändå}
\index[anfa]{Vi klarar oss nog ändå}
\songinfo{Text: Lasse Dahlquist}

\begin{parse lines}[\noindent]{#1\\}
    Jag vill sjunga en visa i klaraste dur,
    ty den handlar om Skåne och slätter och djur.
    Kan hända den retar en del men i så fall är det deras eget fel.
    Det har talats så mycket om dynga och lort men betänk vilken oerhörd nytta den gjort.
    Så låt dom bara gå på vi klarar oss nog ändå, Ja låt dom bara gå på vi klarar oss nog ändå

    Kanske språket vi talar ej klingar så väl,
    men det är och förbliver en del av vår själ.
    Kan hända det retar…
    Uti självaste riksdagen på skånska dom slåss,
    för de flesta utav dom har kommit från oss.
    Så låt dom bara gå på...

    Utav våra produkter de smörjer sitt krås,
    och de är ifrån oss som dom fått Mårten Gås.
    Kan hända det retar…
    Det har klagats på bostaden på våra svin
    men när julskinkan kommer jo då är den fin.
    Så låt dom bara gå på...
\end{parse lines}

\begin{parse lines}[\noindent]{#1\\}
    Hela landet får njuta av vår akvavit.
    Sockerbetan har lärt dom att dricka på bit.
    Kan hända det retar…
    Våran sandstrand den är både bländvit och fin
    och så har vi ju vårlilla vida kanin.
    Så låt dom bara gå på...

    Selma Lagerlöv som är en fin gammal dam,
    med Nils Holgersson gjorde för Skåne reklam.
    Kan hända det retar…
    Tänk sån nytta som storken från Skåne har gjort,
    men det hindrar ju inte att folk pratar lort.
    Men låt dem bara gå på vi klarar oss nog ändå.
\end{parse lines}

\vissteduatt{Visste du att sjunges på skånska?...}

% \newpage