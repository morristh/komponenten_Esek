\begin{center}
    \vspace*{1.5cm}
    {\fontsize{20}{20}\textbf{Klassiska visor}}\\
    \vspace{0.7cm}
    {\fontsize{12}{12}\textit{Om Taube själv får välja}}
\end{center}
\addtocwithheader{Klassiska visor}  % Add entry to TOC and set header
\thispagestyle{empty}
\noBackground

\newpage
\resetBackground

\subsection*{Måltidssången} 
\index[alfa]{Måltidssången}
\index[anfa]{Så lunka vi så småning om}
\songinfo{C. M. Bellman \\Fredmans sång N:o 21}

\begin{parse lines}[\noindent]{#1\\}
    Så lunka vi så småningom,
    från Bacchi buller och tumult
    när döden ropar: Granne, kom,
    ditt timglas är nu fullt
    Du gubbe fäll din krycka ner
    och du, du yngling lyd min lag:
    den skönsta nymf som åt dig ler
    inunder armen tag

    Tycker du att graven är för djup,
    nå välan, så tag dig då en sup,
    tag dig sen dito en, dito två, dito tre
    så dör du nöjdare

    Säg, är du nöjd, min granne säg?
    Så prisa värden nu till slut
    om vi har en och samma väg,
    så följoms åt... Drick ut!
    Men först med vinet rött och vitt,
    för vår värdinna bugom oss
    och halkom sen i graven fritt
    vid aftonstjärnans bloss

    Tycker du…
\end{parse lines}

\vissteduatt{Visste du att denna sjunges när alla fått sin varmrätt? %\\ När man sjunger "tycker du..." ska man kroka i arm med sina bordsgrannar    ?\\}

\newpage

\subsection*{Den blomstertid nu kommer} 
\index[alfa]{Den blomstertid nu kommer}
\index[anfa]{Den blomstertid nu kommer}

\begin{parse lines}[\noindent]{#1\\}
    Den blomstertid nu kommer
    med lust och fägring stor
    Nu nalkas ljuvlig sommar
    då gräs och gröda gror
    Den blida solen väcker
    allt det som varit dött
    Den allt med grönska täcker,
    och allt blir återfött

    De fagra blomsterängar
    och åkerns ädla säd,
    de rika örtesängar
    och alla gröna träd
    skall oss var dag påminna
    Guds godhets rikedom
    Låt oss den nåd besinna
    som räcker året om

\end{parse lines}

\newpage

\subsection*{Studentsången} 
\index[alfa]{Studentsången}
\index[anfa]{Sjung om studentens lyckliga dag}
\songinfo{Musik: Prins Gustaf\\
Text: Herman Sätherberg
}

\begin{parse lines}[\noindent]{#1\\}
    Sjung om studentens lyckliga dag,
    låtom oss fröjdas i ungdomens vår!
    Än klappar hjärtat med friska slag,
    och den ljusnande framtid är vår.
    ||: Inga stormar än
    i våra sinnen bo,
    hoppet är vår vän,
    och vi dess löften tro,
    när vi knyta förbund i den lund,
    där de härliga lagrarna gro!
    där de härliga lagrarna gro! :||
    Hurra!

\end{parse lines}

\newpage

\subsection*{O, gamla klang och jubeltid} 
\index[alfa]{O, gamla klang och jubeltid}
\index[anfa]{O, gamla klang och jubeltid}
\songinfo{Mel: O, alte Burschenherrlichkeit!
}

\begin{parse lines}[\noindent]{#1\\}
    O, gamla klang och jubeltid, 
    ditt minne skall förbliva
    och än åt livets bistra strid 
    ett rostigt skimmer giva!
    Snart tystnar allt vårt yra skämt,
    vår sång blir stum, vårt glam förstämt;
    o jerum, jerum, jerum,
    o, quae mutatio rerum!

    Var äro de, som kunde allt,
    blott ej sin ära svika,
    som voro män av äkta halt
    och världens herrar lika?
    De drogo bort från vin och sång
    till vardagslivets tråk och tvång;
    o, jerum, jerum, jerum,
    o, quae mutatio rerum!

    En tämjer forsens vilda fall,
    en annan ger oss papper,
    en idkar maskinistens kall,
    en mästrar volt så tapper,
    en ritar hus, en mäter mark,
    en blandar hop mixtur så stark;
    o, jerum, jerum, jerum,
    o, quae mutatio rerum!
\end{parse lines}

\vissteduatt{Visste du att denna sången sjungs när en sittning ska avlutas på\\ andra sektioner?}

\begin{parse lines}[\noindent]{#1\\}
    Men hjärtat i en sann student
    kan ingen tid förfrysa,
    den glädjeeld, som där han har tänt,
    hans hela liv skall lysa.
    Det gamla skalet brustit har,
    men kärnan finnes frisk dock kvar,
    och vad han än må mista,
    den skall dock aldrig brista!

    Så sluten, bröder, fast vår krets
    till glädjens värn och ära!
    Trots allt vi tryggt och väl tillfreds
    vår vänskap trohet svära.
    Lyft bägarn högt, och klinga, vän!
    De gamla gudar leva än
    bland skålar och pokaler,
    bland skålar och pokaler!
\end{parse lines}

\vissteduatt{Visste du att den tredje versen är LTH:s och KTH:s egna vers?}

\newpage

\subsection*{Längtan till landet} 
\index[alfa]{Längtan till landet}
\index[anfa]{Vintern rasat ut bland våra fjällar}
\songinfo{Musik: O. Lindblad\\ Text: H. Sätherberg}

\begin{parse lines}[\noindent]{#1\\}
    Vintern rasat ut bland våra fjällar,
    drivans blommor smälta ned och dö
    Himlen ler i vårens ljusa kvällar
    solen kysser liv i skog och sjö
    ||: Snart är sommar´n här! I purpurvågor,
    guldbelagda, azurskiftande
    ligga ägnarne i dagens lågor
    och i lunden dansa källorne :||

    Ja, jag kommer! Hälsen glada vindar
    ut till landet, ut till fåglarne,
    att jag älskar dem, till björk och lindar,
    sjö och berg, jag vill dem återse
    ||: Se dem än som i min bardoms stunder
    följa bäckens dans till klarnad sjö
    Trastens sång i furuskogens lunder,
    vattenfågelns lek kring fjärd och ö :||
\end{parse lines}

\vissteduatt{Visste du att sången sjungs av Lunds studentsångare 1:a maj varje\\år för att fira in våren?}

\newpage

\subsection*{Du gamla, du fria} 
\index[alfa]{Du gamla, du fria}
\index[anfa]{Du gamla, du fria}
\songinfo{Text: Richard Dybeck}

\begin{parse lines}[\noindent]{#1\\}
    Du gamla, du fria, du fjällhöga Nord
    du tysta, du glädjerika sköna
    Jag hälsar dig vänaste land uppå jord
    Din sol, din himmel, dina ängder gröna
    Din sol, din himmel, dina ängder gröna

    Du tronar på minnen från fornstora dar
    då ärat ditt namn flög över jorden
    Jag vet att du är och du blir vad du var
    Ja, jag vill leva, jag vill dö i Norden
    Ja, jag vill leva, jag vill dö i Norden
\end{parse lines}

% \vissteduatt{visste du att...En händelse som enligt historien bidrog till \\
% att sången började få status som nationalsång var vid en promotionsmiddag \\
% vid Lunds Universitet våren 1893, då Kung Oscar II ställde sig upp när sången framfördes.}

\vissteduatt{Visste du att enligt historien fick sången status som nationalsång\\efter en promotionsmiddag vid Lunds Universistet 1893, när\\Kung Oscar II reste sig under framförandet?}

\newpage

\subsection*{Brevet från kolonien} 
\index[alfa]{Brevet från kolonien}
\index[anfa]{Brevet från kolonien}
\songinfo{Cornelis Vreeswijk}

\colorbox{yellow}{OSÄKER PÅ DENNA}

\begin{parse lines}[\noindent]{#1\\}
    Hejsan morsan, hejsan stabben
    Här är brev från älsklingsgrabben
    Vi har kul på kolonien
    Vi bor tjugoåtta gangstergrabbar i en

    Stor barack med massa sängar
    Kan ni skicka mera pengar?
    För det vore en god gärning
    Jag har spelat bort vartenda dugg på tärning

    Här är roligt vill jag lova
    Fastän lite svårt att sova
    Killen som har sängen över mig
    Han vaknar inte han när han behöver, nej

    Jag har tappat två framtänder
    För jag skulle gå på händer
    När vi lattjade charader
    Så när morsan nu får se mig får hon spader

    Ute i skogen finns baciller
    Men min kompis han har piller
    Som han köpt utav en ful typ
    Och om man äter dem blir man en jättekul typ

    Våran fröken är försvunnen
    Hon har dränkt sig uti brunnen
    För en morgon blev hon galen
    När vi släppte ut en huggorm i matsalen

    Men jag är inte, rädd för spöken
    För min kompis han har kröken
    Som han gjort utav potatis
    Och som han säljer i baracken nästan gratis

    Föreståndaren han har farit
    Han blir aldrig var han varit
    För polisen kom och tog hand
    Om honom förra veckan när vi lekte skogsbrand

    Ute i skogen finns det rådjur
    I baracken finns det smådjur
    Och min bäste kompis Tage
    Han har en liten fickkniv inuti sin mage

    Honom ska de operera
    Ja, nu vet jag inge mera
    Kram och kyss och hjärtligt tack sen
    Men nu ska vi ut och bränna grannbaracken
\end{parse lines}

\vissteduatt{visste du att...}

\newpage

\subsection*{Skånska slott och herresäten} 
\index[alfa]{Skånska slott och herresäten}
\index[anfa]{Skånska slott och herresäten}
\songinfo{Text: Hjalmar Gullberg, Bengt Hjelmqvist,
Sång: Edvard Persson}

\colorbox{yellow}{OSÄKER PÅ DENNA}

\begin{parse lines}[\noindent]{#1\\}
    På himmelen vandra sol, stjärnor och måne
    Och kasta sitt fagraste ljus över Skåne
    På höga och låga, på stort och på smått
    På statarens koja och ädlingens slott

    Se månstrålen in genom blyrutan faller
    Och tecknar på golvet det järnsmidda galler
    Stolts jungfrun hon drömmer i majnattens ljus
    Att friare komma till Glimmingehus

    På utflykt till Bokskogen Malmöbon glor upp
    Mot raden av strålande fönster på Torup
    Att smaka på kaka som bakats på spett
    Dig ber hennes nåd, friherrinnan Coyet

    Där rådjuren skymta bak'vitgråa stammar
    Man ser Toppela'gård med broar och dammar
    Systemet på sprit och på skatterna sta'n
    Där lurar belåtet fiskalen Aschan

    Med port genom huset och gamla kanaler
    Lyss Skabersjö ännu till jaktens signaler
    Själv kungen i nåder far dit från sitt slott
    Och skjuter fasaner med grevarna Thott

    Och därefter hälsar han på baron Trolle
    Och jagar och spelar sin sans och sin nolle
    Allt medan baronens gemål
    Plockar gräs åt rastupp och rashöna på Trollenäs
\end{parse lines}

\vissteduatt{visste du att...}


\newpage

\subsection*{Kungssången} 
\index[alfa]{Kungssången}
\index[anfa]{Ur svenska hjärtans djup en gång}
\songinfo{Musik: Otto Lindblad Text: C.V.A Strandberg}

\colorbox{yellow}{OSÄKER PÅ DENNA}

\begin{parse lines}[\noindent]{#1\\}
    Ur svenska hjärtans djup en gång en samfälld och en enkel sång,
    som går till kungen fram! Var honom trofast, och hans ätt,
    gör kronan på hans hjässa lätt, och all din tro till honom sätt, 
    du folk av frejdad stam! O konung, folkets majestät är även ditt: 
    beskärma det och värna det från fall! Stå oss all världens härar mot, 
    vi blinka ej för deras hot: vi lägga dem inför din fot - en kunglig fotapall. 
    Du himlens Herre, med oss var, som förr Du med oss varit har, 
    och liva på vår strand det gamla lynnets art igen hos sveakungen och hans män. 
    Och låt Din ande vila än utöver nordanland!
\end{parse lines}

\vissteduatt{visste du att vissa studenter inleder alla sittningar med denna...}


\newpage


