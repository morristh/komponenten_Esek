
Borra ett hål i pärmen här för att fästa din penna: {\Huge $\bullet$}

\subsubsection*{Konsten att sjunga!}
Att sjunga är verkligen något av det roligaste som finns\dots


Jag vet inte riktigt vad vi vill skriva här\dots

Att sjunga är kul\dots



Morris Thånell BME19 och Elin Helmersson E21\\
Sångbokskommittén 2024



\newpage

Tack till de som hjälpt oss med boken!

\newpage

\begin{center}
    \textbf{Viktig information}
\end{center}

% \backgroundsetup{ 
%   scale=0.36,
%   opacity=1,
%   angle=0,
%   color=black,
%   vshift=300,
%   hshift=151,
%   contents={\includegraphics[width=\paperwidth]{./bilder/profilbild.png}}
% }

\begin{textblock*}{6cm}(6cm,1.5cm) % {block width}(x-coordinate, y-coordinate)
  \includegraphics[width=3.5cm]{./bilder/profilbild_stor.png} % Adjust the image size as needed
\end{textblock*}
\begin{parse lines}[\noindent]{#1\\}
    Ägare:

    Sektion(?):
    
    Inskrivningsår:
    
    Födelsedatum:



    Lämna tillbaka mig till den här adressen:


    Hittelön:
    Om jag inte varit teknolog hade jag varit:
    Om jag fick välja nollningstema:


    Favoritmat:
    Favoritband:
    Favorit
    Bästa 
    Favoritintegral:
    
    .... Rita eller klistra en bild

\end{parse lines}

\newpage

% \subsubsection*{Innehållsförteckning}
\tableofcontents

\newpage
\begin{center}
  \vspace*{1.5cm}
  {\fontsize{20}{20}\textbf{Vett och etikett}}\\
  \vspace{0.7cm}
  {\fontsize{12}{12}\textit{Om den pryde själv får välja}}
\end{center}
\addcontentsline{toc}{section}{Vett och etikett}
\noBackground
\newpage

\subsubsection*{KLÄDKOD}
För att underlätta förmedlingen av vem som ska ha på sig vad använder vi oss av följande namn.

\textbf{Truls}: Manlig teknolog\\
\textbf{Trula}: Kvinnlig teknolog\\
\textbf{Trelsa}: För hen som inte vill identifiera sig med någon av ovanstående.\\

\subsubsection*{Högtidsdräkt - Militäruniform och folkdräkt}

\textbf{Truls}: Militäruniform, folkdräkt eller frack\\
\textbf{Trula}: Militäruniform, folkdräkt eller balklädding\\
\textbf{Trelsa}: Se Truls/Trula\\

\subsubsection*{Civil högtidsdräkt}

\textbf{Truls}: Frack, vit skjorta, vit fluga\\
\textbf{Trula}: Balklädding\\
\textbf{Trelsa}: Se Truls/Trula\\

\subsubsection*{Smoking}
\textbf{Truls}: Smoking, vit skjorta, svart fluga\\
\textbf{Trula}: Lång klänning, dock behöver den inte vara lika elegant som en Balklädding. Tänk festligt.
\textbf{Trelsa}: Se Truls/Trula\\

\subsubsection*{Mörk kostym}
\textbf{Truls}: Mörkblå, mörkgrå eller svart kostym. Vit skjorta med sidenslips eller fluga i valfri färg.\\
\textbf{Trula}: En finare känning, men även byxdress elelr halvlång kjol med jacka går bra.\\
\textbf{Trelsa}: Se Truls/Trula\\

\subsubsection*{Kavaj}
Ibland även kallad bruten elelr udda kavaj.\\
\textbf{Truls}: Kavaj och ett par finare byxor (dock inte kostymbyxor), skjorta i valfri färg. Fluga eller slips kan vara trevligt!\\
\textbf{Trula}: Cocktailklänning, kjol eller dress.\\
\textbf{Trelsa}: Se Truls/Trula\\

\subsubsection*{Mörk kostym}
\textbf{Truls}: Ouvve \textbf{Hur vill vi stava ouvve/ovve i boken?}\\
\textbf{Trula}: Ouvve\\
\textbf{Trelsa}: Se Truls/Trula\\


\subsubsection*{KLÄDKOD version 2?}
Kåren har ett annat upplägg på hur de visar klädkoderna. Vill vi göra som dem?
\\

Exempel (kopierat från kårens bok)

\subsubsection*{Högtidsdräkt/högtidsklädsel}
För honom är frack lämplig. 
En fin golvlång klänning gäller för henne 
- ett bra material och ett tjusigt sntit ska det vara på den. 
Vill man bära handskar till klänningen går det bra, 
men glöm inte att ta av dem när du äter. 
Både han och hon kan istälet bära hembygdsdräkt eller militär högtidsdräkt.


\newpage

\subsubsection*{ETIKETT}

\subsubsection*{Vid bordet}
Han har sin bordsdam till höger om sig och hon har sin bordsherre till vänster.
Herren drar ut stolen till höger för att hjälpa sin bordsdam till bords eller från bordet. 


\textbf{Denna visste du att måste nog skrivas om eller ändra storleken på all text i boken. Vi ahr lite större text än i gamla komponenten vilket göra att vi inte får plats med riktigt lika mycket text på varje sida....}

\vissteduatt{Visste du att bordsherre och bordsdam är bordsplaceringsbenämningar\\ som avser att underlätta sittningsförfarande och är helt könsneutralt?}