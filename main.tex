\documentclass{article}
\usepackage[a6paper,
            total={105mm, 148mm},
            margin=30pt,
            twoside % Tillåter olika placering av sidnummer på udda och jämna sidor
]{geometry}
\usepackage{fancyhdr}
\usepackage{parselines}
\usepackage{float}
\usepackage{graphicx}
\usepackage[absolute,overlay]{textpos}  % Package for absolute positioning
\usepackage{atbegshi}

\usepackage{imakeidx} % Registret
\usepackage{background} % Gråa Krusidull-E:et och alla bilder
\usepackage{ifoddpage} % Gråa Krudidull-E:et på udda sidor


\usepackage{xcolor} % BARA I BÖRJAN FÖR MARKERING TILL OSS SJÄLVA


% Normal background setup for odd pages
\backgroundsetup{
  scale=0.65,
  opacity=0.075,
  angle=0,
  color=black,
  vshift=-130,
  hshift=50,
  contents={%
    \checkoddpage
    \ifoddpage
      \includegraphics[width=\paperwidth]{./bilder/large_E.png}
    \else
      %\includegraphics[width=\paperwidth]{../bilder/VarumarkesBildServlet.jpg}
    \fi
  }
}


\newcommand{\custombackground}[1]{
  \backgroundsetup{
    scale=0.65,
    opacity=0.0,
    angle=0,
    color=black,
    vshift=-130,
    hshift=50,
    contents={\includegraphics[width=\paperwidth]{#1}}
  }
}

\newcommand{\noBackground}{
  \backgroundsetup{
    scale=0.65,
    opacity=0.0,
    angle=0,
    color=black,
    vshift=-130,
    hshift=50,
    contents={\includegraphics[width=\paperwidth]{./bilder/no_background.png}}
  }
}


\newcommand{\resetBackground}{
  \backgroundsetup{
    scale=0.65,
    opacity=0.075,
    angle=0,
    color=black,
    vshift=-130,
    hshift=50,
    contents={%
      \checkoddpage
      \ifoddpage
        \includegraphics[width=\paperwidth]{./bilder/large_E.png}
      \else
        %\includegraphics[width=\paperwidth]{../bilder/VarumarkesBildServlet.jpg}
      \fi
    }
  }
}



% Disable indentation for the whole document
\setlength{\parindent}{0pt}


% Use fontspec with XeLaTeX or LuaLaTeX to load system fonts
\usepackage{fontspec}

% Set up the subsection font to Lucida Sans Unicode
\usepackage{titlesec}
\titleformat{\subsection}
  {\normalfont\fontsize{12pt}{10pt}\selectfont\fontspec{Lucida Sans Unicode}} % Adjust the size if needed
  {\thesubsection}{1em}{}

% Adjust vertical spacing around subsection titles
\titlespacing{\subsection}
{0pt}                % Left margin
{0.5ex plus .2ex}    % Space before subsection title (adjust as needed)
{0.5ex plus .2ex}    % Space after subsection title (adjust to decrease space after)

% Adjust footskip to add space below the footer
% \setlength{\footskip}{20pt} % Flyttar upp footern lite ########################################## Denna funkar inte riktigt som jag trodde. Jag vill flytta upp sidnumret lite mer
\setlength{\textheight}{120mm}


% Define the page style
\fancypagestyle{main}{
  \fancyhf{} % clear all header and footer fields
  \fancyhead[RO]{\hfill \footnotesize\scshape\leftmark \hfill}
  % \fancyfoot[LE,RO]{\thepage}

  % Adjust the page number position
  % Set page number closer to the edge by increasing the right margin
  \fancyfoot[LE]{\hspace{-17pt}\thepage} % Flytta ut sidnummer jämna sidor
  \fancyfoot[RO]{\thepage\hspace{-17pt}} % Flytta ut sidnummer udda sidor

  % Ta bort linje under/över header och footer
  \renewcommand{\headrulewidth}{0 pt}
  \renewcommand{\footrulewidth}{0 pt}

  % Använder footnote for 'visste du att'
  \renewcommand{\footnoterule}{}
  \renewcommand{\thefootnote}{}
}

% Set up textblock package to work with A6 paper
\setlength{\TPHorizModule}{1mm}  % Set horizontal units to mm
\setlength{\TPVertModule}{1mm}   % Set vertical units to mm

% Define the custom command to place text at the bottom of the page
\newcommand{\vissteduatt}[1]{%
    \begin{textblock*}{100mm}(12mm,137mm) % Adjust (2.5mm, 140mm) to position text
        \raggedright
        {\footnotesize{\textit{#1}}}
    \end{textblock*}
    \newpage % Ensures that the text appears only on the current page
}


% Define the custom command for small font and italics
\newcommand{\songinfo}[1]{%
  \textit{\small #1 \\}%
}









% DETTA SÄTTET ATT GÖRA REGISTER FUNGERAR BRA OCH ÄR LÄTT MEN DET BLIR INTE LIKA SNYGGT UTAN PUNKTERNA



% HITTA ETT SÄTT ATT GÖRA ETT SNYGGARE REGISTER INNAN VI FORTSÄTTER ATT LÄGGA TILL LÅTAR ETC. 
\makeindex[ % Alfabetiskt register
  name=alfa,
  columns=1,
  title=Alfabetiskt Register,
  intoc,
  options = {-s styleAlfa.ist}
]
\makeindex[ % Analfabetiskt register -- början av sången
  name=anfa,
  columns=1,
  title=Analfabetiskt Register,
  intoc,
  options = {-s styleAlfa.ist}
]


\begin{document}

% Apply the 'main' page style
\pagestyle{main}

% Title and empty page without header/footer
\NoBgThispage
\begin{titlepage}
    \centering
    \vspace{1cm}
    {\fontsize{18}{18}\selectfont E-sektionens sångbok}\\
    \vspace{0.2cm}
    {\fontsize{30}{30}\textbf{Komponenten}}\\
    \vspace{0.2cm}
    {\fontsize{15}{15}\textit{2:a upplagan}}
    \thispagestyle{empty}

    \begin{figure}[H]
      \centering
      \includegraphics[width=1\textwidth]{./bilder/large_E.png}
    \end{figure}

\end{titlepage}


\newpage


Borra ett hål i pärmen här för att fästa din penna: {\Huge $\bullet$}

\subsubsection*{Konsten att sjunga!}
Att sjunga är verkligen något av det roligaste som finns\dots


Jag vet inte riktigt vad vi vill skriva här\dots

Att sjunga är kul\dots



Morris Thånell BME19 och Elin Helmersson E21\\
Sångbokskommittén 2024



\newpage

Tack till de som hjälpt oss med boken!

\newpage

\begin{center}
    \textbf{Viktig information}
\end{center}

% \backgroundsetup{ 
%   scale=0.36,
%   opacity=1,
%   angle=0,
%   color=black,
%   vshift=300,
%   hshift=151,
%   contents={\includegraphics[width=\paperwidth]{./bilder/profilbild.png}}
% }

\begin{textblock*}{6cm}(6cm,1.5cm) % {block width}(x-coordinate, y-coordinate)
  \includegraphics[width=3.5cm]{./bilder/profilbild_stor.png} % Adjust the image size as needed
\end{textblock*}
\begin{parse lines}[\noindent]{#1\\}
    Ägare:

    Sektion(?):
    
    Inskrivningsår:
    
    Födelsedatum:



    Lämna tillbaka mig till den här adressen:


    Hittelön:
    Om jag inte varit teknolog hade jag varit:
    Om jag fick välja nollningstema:


    Favoritmat:
    Favoritband:
    Favorit
    Bästa 
    Favoritintegral:
    
    .... Rita eller klistra en bild

\end{parse lines}

\newpage

% \subsubsection*{Innehållsförteckning}
\tableofcontents

\newpage
\begin{center}
  \vspace*{1.5cm}
  {\fontsize{20}{20}\textbf{Vett och etikett}}\\
  \vspace{0.7cm}
  {\fontsize{12}{12}\textit{Om den pryde själv får välja}}
\end{center}
\addcontentsline{toc}{section}{Vett och etikett}
\noBackground
\newpage

\subsubsection*{KLÄDKOD}
För att underlätta förmedlingen av vem som ska ha på sig vad använder vi oss av följande namn.

\textbf{Truls}: Manlig teknolog\\
\textbf{Trula}: Kvinnlig teknolog\\
\textbf{Trelsa}: För hen som inte vill identifiera sig med någon av ovanstående.\\

\subsubsection*{Högtidsdräkt - Militäruniform och folkdräkt}

\textbf{Truls}: Militäruniform, folkdräkt eller frack\\
\textbf{Trula}: Militäruniform, folkdräkt eller balklädding\\
\textbf{Trelsa}: Se Truls/Trula\\

\subsubsection*{Civil högtidsdräkt}

\textbf{Truls}: Frack, vit skjorta, vit fluga\\
\textbf{Trula}: Balklädding\\
\textbf{Trelsa}: Se Truls/Trula\\

\subsubsection*{Smoking}
\textbf{Truls}: Smoking, vit skjorta, svart fluga\\
\textbf{Trula}: Lång klänning, dock behöver den inte vara lika elegant som en Balklädding. Tänk festligt.
\textbf{Trelsa}: Se Truls/Trula\\

\subsubsection*{Mörk kostym}
\textbf{Truls}: Mörkblå, mörkgrå eller svart kostym. Vit skjorta med sidenslips eller fluga i valfri färg.\\
\textbf{Trula}: En finare känning, men även byxdress elelr halvlång kjol med jacka går bra.\\
\textbf{Trelsa}: Se Truls/Trula\\

\subsubsection*{Kavaj}
Ibland även kallad bruten elelr udda kavaj.\\
\textbf{Truls}: Kavaj och ett par finare byxor (dock inte kostymbyxor), skjorta i valfri färg. Fluga eller slips kan vara trevligt!\\
\textbf{Trula}: Cocktailklänning, kjol eller dress.\\
\textbf{Trelsa}: Se Truls/Trula\\

\subsubsection*{Mörk kostym}
\textbf{Truls}: Ouvve \textbf{Hur vill vi stava ouvve/ovve i boken?}\\
\textbf{Trula}: Ouvve\\
\textbf{Trelsa}: Se Truls/Trula\\


\subsubsection*{KLÄDKOD version 2?}
Kåren har ett annat upplägg på hur de visar klädkoderna. Vill vi göra som dem?
\\

Exempel (kopierat från kårens bok)

\subsubsection*{Högtidsdräkt/högtidsklädsel}
För honom är frack lämplig. 
En fin golvlång klänning gäller för henne 
- ett bra material och ett tjusigt sntit ska det vara på den. 
Vill man bära handskar till klänningen går det bra, 
men glöm inte att ta av dem när du äter. 
Både han och hon kan istälet bära hembygdsdräkt eller militär högtidsdräkt.


\newpage

\subsubsection*{ETIKETT}

\subsubsection*{Vid bordet}
Han har sin bordsdam till höger om sig och hon har sin bordsherre till vänster.
Herren drar ut stolen till höger för att hjälpa sin bordsdam till bords eller från bordet. 


\textbf{Denna visste du att måste nog skrivas om eller ändra storleken på all text i boken. Vi ahr lite större text än i gamla komponenten vilket göra att vi inte får plats med riktigt lika mycket text på varje sida....}

\vissteduatt{Visste du att bordsherre och bordsdam är bordsplaceringsbenämningar\\ som avser att underlätta sittningsförfarande och är helt könsneutralt?}

\begin{center}
  \vspace*{1.5cm}
  {\fontsize{20}{20}\textbf{Testlåtar}}\\
  \vspace{0.7cm}
  {\fontsize{12}{12}\textit{Om Morris och Elin själva får välja}}

\end{center}
\addcontentsline{toc}{section}{Testlåtar}
\noBackground

\newpage
\resetBackground



\subsection*{Vikingen}
\index[alfa]{Vikingen}
\index[anfa]{En viking}
\songinfo{Mel: When Johnny comes marching home\\
          E-sektionen Sångarstriden 1981}

\begin{parse lines}[\noindent]{#1\\}
En viking vill ha livets vann,
hurra, hurra!
Den hastigt i mitt svalg försvann,
hurra, hurra!
Till kalv, till oxe, till fisk, till fläsk,
när somliga bara dricker läsk,
då vill alla sanna vikingar ha en bäsk.

När vi druckit bäsken slut,
tragik, tragik!
Då bäres varje viking ut,
som lik sig lik.
Och sen, om vi vaknar, vi sjunger en bit,
sen korkar vi upp Skånes akvavit.
Skål för alla vikingar som kom hit!

\end{parse lines}

\vissteduatt{Visste du att... kul info on Vikingen. hej ehj Här är texten nu \\ lite längre än tidigare kul va? }

\newpage

\subsection*{Portos visa}
\index[alfa]{Portos visa}
\index[anfa]{Jag vill ut och gasqua}
\songinfo{Mel: You can't get a man with a gun}\\
\begin{parse lines}[\noindent]{#1\\}

Jag vill ut och gasqua!
Var fan är min flaska?
Vem i helvete stal min butelj?
Skall mej törsten betvinga?
En TT börja svinga?
Nej för fan bara blunda och svälj!
Vilken smörja!
Får jag spörja?
Vem för fan tror att jag är en älg?

Till England vi rider,
och sedan vad det lider
träffar vi välan på någon PUB.
Och där ska vi festa!
Blott dricka av det bästa
utav whiskey och portvin.
Jag tänker gå hårt in
för att prova på rubb och stubb.
\end{parse lines}

\vissteduatt{Visste du att... portos visa}


\newpage

\subsection*{Jag ska festa}
\index[alfa]{Jag ska festa}
\index[anfa]{Jag ska festa}
\songinfo{Mel: Bamse\\ D-sektionen Sångarstriden 1987}
\begin{parse lines}[\noindent]{#1\\}

Jag ska festa, ta det lugnt med spriten,
ha det roligt utan å va full.
Inte krypa runt med festeliten,
ta det varligt för min egen skull

Först en öl i torra strupen,
efter det så kommer supen,
i med vinet, ner med punschen.
Sist en groggbuffé.

Jag är skitfull, däckar först av alla,
missar festen, men vad gör väl de?
Blandar hejdlöst öl och gammal filmjölk,
kastar upp på bordsdamen breve!

Sen en öl...

Spyan rinner ner för ylleslipsen.
Raviolin torkar i mitt hår.
Vem har lagt mig under matsalsbordet?
Vems är gaffeln i mitt högra lår?

Sist en öl...
\end{parse lines}
\vissteduatt{Visste du att? något...}

\newpage

\backgroundsetup{
  scale=0.65,
  opacity=0.2,
  angle=0,
  color=black,
  vshift=-130,
  hshift=50,
  contents={\includegraphics[width=\paperwidth]{./bilder/VarumarkesBildServlet.jpg}}
}
\subsection*{Everywhere we go}
\songinfo{Mel: Everywhere we go}
\index[alfa]{Everywhere we go}
\index[anfa]{Everywhere we go...}
\begin{parse lines}[\noindent]{#1\\}

Everywhere we go
People wanna know
Who we are
So we tell them

We are the E-sek
Mighty, mighty E-sek

$\vert\vert$: Uh, ah, oh E-sek :$\vert\vert$ (3 ggr)\\
\end{parse lines}

\subsection*{Vår färg är röd}
\songinfo{Mel: When the saints go marching in}
\index[alfa]{Vår färg är röd}
\index[anfa]{Vår färg är röd}
\begin{parse lines}[\noindent]{#1\\}

Vår färg är röd, vår färg är fin,
för det är vi som går Maskin
Och vi har kommit för att dricka alkohol,
för det är vi som går Maskin
\end{parse lines}

\newpage

\resetBackground

\newpage
\resetBackground




\begin{center}
    \vspace*{1.5cm}
    {\fontsize{20}{20}\textbf{E-sektionens kampvisor}}\\
    \vspace{0.7cm}
    {\fontsize{12}{12}\textit{Om E:aren själv får välja}}  
\end{center}
\addtocwithheader{E-sektionens kampvisor}  % Add entry to TOC and set header
\noBackground

\newpage
\resetBackground


\subsection*{E-sektionens Kampvisa I}
\index[alfa]{E-sektionens Kampvisa I}
\index[anfa]{Det sprakar så glatt i vårt hår}
\songinfo{Mel: Starts and Stripe \textbf{OSÄKER MELODI?}}

\begin{parse lines}[\noindent]{#1\\}
    Det sprakar så glatt i vårt hår
    ur öronen sprutar det gnistor
    och strömmarna i våra tår
    laddar upp oss när vi går.

    I hjärnan vi har resistans
    och ström genom motstånd ger en spänning
    och spänning det vill vi ju ha.
    Vi går på E! Vi går på E!
    Vi går på Elström!

\end{parse lines}


\subsection*{E-sektionens Kampvisa II} 
\index[alfa]{E-sektionens Kampvisa II}
\index[anfa]{E, E, E vill vi se}
\songinfo{Mel: Trink, trink, brüderlein trink}

\begin{parse lines}[\noindent]{#1\\}
    E, E, E vill vi se
    E är det bästa där é
    E, E, E vill vi se
    E får oss alla att le!
    Spänning där é i vär erotik
    vi kör med medicin och teknik.
    Spänning där é i vår erotik
    vi kör med elektroteknik

\end{parse lines}

\vissteduatt{Visste du att E-sektionen har sjukt många kampvisor?}

% \newpage

\subsection*{E-sektionens Kampvisa III} 
\index[alfa]{E-sektionens Kampvisa III}
\index[anfa]{Vi é elteknister ifrån LTH}
\songinfo{Mel: Vi äro musikanter}

\begin{parse lines}[\noindent]{#1\\}
    Vi é elteknister ifrån LTH.
    Hos oss finns inga brister och dé é ju som så.

    (Att) Vi kan lödda, räkna, dricka, bränna sprit.
    Upp till Lophtet, kom så går vi dit!

    Vi kan räkna tangens, sinus, derivera, integraler.
    Vi kan koda singelchip och mata våran pic.

    Ett in, noll ut, rulla kabel, heja vit!
    Vi vill att just du skall komma hit.

    Hoppa i din vita overall och börja gå.
    Elteknik på LTH det bästa du kan få!

\end{parse lines}


\subsection*{E-sektionens Kampvisa IV} 
\index[alfa]{E-sektionens Kampvisa IV}
\index[anfa]{Vi vill ha mera E}
\songinfo{Mel: Mera mål}

\begin{parse lines}[\noindent]{#1\\}
    Vi vill ha mera E, flera E
    Å E det kommer ni att se
    Vi vill ha mera E, mycket mer
    Se upp här kommer vi från E

\end{parse lines}

\vissteduatt{Visste du att E-sektionen är den äldsta sektionen på LTH?}

\newpage


\subsection*{E-sektionens Kampvisa V} 
\index[alfa]{E-sektionens Kampvisa V}
\index[anfa]{Alla vi på E-sek klappar nu} % VILL VI HA DENNA ENS?
\songinfo{Mel: Klappa händerna}

\begin{parse lines}[\noindent]{#1\\}
    //: Alla vi som går på E-sek klappar nu ://

\end{parse lines}


\subsection*{E-sektionens Kampvisa VI} 
\index[alfa]{E-sektionens Kampvisa VI}
\index[anfa]{Vi går på E-sek}
\songinfo{Mel: Man ska ha husvagn}

\begin{parse lines}[\noindent]{#1\\}
    Vi går på E-sek, och vi har spänning så som få
    Vi går på E-sek, äldst och bäst på LTH
    Vi går på E-sek, vår färg är vit, Elektrovit!
    Vi går på E-sek, LTH:s elit!

\end{parse lines}

\vissteduatt{Visste du att... }


\newpage

\subsection*{E-sektionens Kampvisa VII} 
\index[alfa]{E-sektionens Kampvisa VII}
\index[anfa]{E-sek på LTH} 
\songinfo{Mel: Anton aus Tirol \\ Text: Sebastian Elm (E12), Kewin Erichsen (E11)}

\begin{parse lines}[\noindent]{#1\\}
    //: E-sek, E-sek, E-sek på LTH ://
    Vi slajdar in,
    med grym entré
    För det är vi som går på E!
    Vi lyser upp med bra manér,
    tänder gnistan inom er
    Det är eliten som ni ser!

    Är du för klen?
    De' e ingen kris
    Det fixar vi med dialys!
    När ni sedan vaknat opp,
    är vår spänning än på topp
    För det är vi som går på E!
    //: E-sek, E-sek, E-sek på LTH ://

\end{parse lines}

\vissteduatt{Visste du att denna kampvisan har en musikvideo på Youtube?}


\newpage

\subsection*{E-sektionens Kampvisa VIII} 
\index[alfa]{E-sektionens Kampvisa VIII}
\index[anfa]{Alla vill höra våran sång} 
\songinfo{Mel: Anton aus Tirol \\ Sebastian Elm E12 \& Kewin Erichsen E11}

\begin{parse lines}[\noindent]{#1\\}
    Alla vill höra våran sång
    Därför vi sjunga gång på gång
    E-sek, här kommer E-sek
    Och alla andra, de kommer igång

\end{parse lines}


\subsection*{E-sektionens Kampvisa IX} 
\index[alfa]{E-sektionens Kampvisa IX}
\index[anfa]{Här kommer E-sektionen} 
\songinfo{Mel: Gärdebylåten}

\begin{parse lines}[\noindent]{#1\\}
    Här kommer E-sektionen
    Vackrast på hela jorden
    Tuborg i Edekvata, billigast i hela Lund
    Med vitt ska vi kiosken måla
    Sen för vår seger skåla
    Festa det gör vi bäst på E
    O alla får va me’
    \textbf{(Om man vill, Och det vill man!)}

\end{parse lines}

\newpage

\subsection*{E-sektionens Kampvisa X} 
\index[alfa]{E-sektionens Kampvisa X}
\index[anfa]{Everywhere we go} 
%\songinfo{Mel: Gärdebylåten}

\begin{parse lines}[\noindent]{#1\\}
    Everywhere we go
    People wanna know
    Who we are
    So we tell them
    We are the E-sek
    Mighty mighty E-sek

    Oh ah å E-sek...

\end{parse lines}


\subsection*{E-sektionens Kampvisa XI} 
% \index[alfa]{E-sektionens Kampvisa X}
% \index[anfa]{Everywhere we go} 
\songinfo{Mel: Coming soon}

\begin{parse lines}[\noindent]{#1\\}
    Coming soon to you sångbok

\end{parse lines}

\vissteduatt{Visste du att du kan skriva nästa sjungbok?/ Du har vad som krävs för...}


\newpage

\subsection*{E-sektionens Sexlåt} 
\index[alfa]{E-sektionens Sexlåt}
\index[anfa]{Sexets låt} 
\songinfo{Mel: När vindarna viskar mitt namn\\
Text: Sebastian Elm E12, Kewin Erichsen E11
}

\begin{parse lines}[\noindent]{#1\\}
    Jag var fångad i sexet, jag såg inget ljus
    In i dimman betygen försvann
    Jag jobbar och sliter, det finns ingen tid
    När FAN låg jag senast i fas

    Men de ger mig min styrka, ger mig sprit att förtära var natt
    En fyllecell, vårt mål, vi trivs bäst i vår vita kavaj!

    Vi mår va några jävla drägg, som går loss och dricker rent
    Men E6 styr upp spänningen

    //: När Gasquen den viskar vårt namn ://

\end{parse lines}

\vissteduatt{Visste du att "Gasquen" kan bytas till aktuell plats där sången framförs?}

\newpage

\subsection*{Medtekingenjören} 
\index[alfa]{Medtekingenjören}
\index[anfa]{Vi som botar alla sjuka} 
\songinfo{Mel: O hur saligt att få vandra
}

\begin{parse lines}[\noindent]{#1\\}
    Vi som botar alla sjuka
    Diagnos med hjälp av ultraljud
    Hårda delar liksom mjuka
    Vi placerar elektroder på din hud
    Och studerar sen effekten
    Tolkar diagrammets amplitud
    Om den påvisar defekter
    Med din hjärna så leker vi gud

    //: En röntgenbild, den gör mig vild
    och jag blir kär, får hjärtbesvär
    En blå pastill, och lite pill
    Får mig att tända till ://

    Gener kan manipuleras
    Ingenjörskonst på ett högre plan
    När vi dig har programmerat
    Blir du aldrig mer likadan
    Läkarna behandlar dig mot gaser
    Rektoskopi ja det slipper vi
    Istället leker vi med laser
    Och röntgen och MRI

\end{parse lines}

\vissteduatt{Visste du att .... något om kth medtech?}

\newpage

\begin{parse lines}[\noindent]{#1\\}
    Och läkarsprit - vår favorit
    Ja läkarsprit - mot hepatit
    Sälj läkarsprit - ge oss profit
    Mer läkarsprit

    Mer läkarsprit - fyll på ge hit
    Jag får aptit - av läkarsprit
    För läkarsprit - vår favorit
    Mer läkarsprit

\end{parse lines}

\vissteduatt{Visste du att .... något mer om kth medtech eller BME?}


\newpage


\subsection*{Mer jul} 
\index[alfa]{Mer jul}
\index[anfa]{Mer jul} 
\songinfo{Av: Falk Adolphson
}

\begin{parse lines}[\noindent]{#1\\}
    Jag är en lugn person med takt och ton
    måttfull och balanserad
    Jag är tyst och still och det ska mycket till 
    innan jag blir exalterad
    Men jag har en last som håller mig fast 
    i ett järngrepp varje vinter
    När året är slut och snön ligger djup 
    och slädarnas medar slinter

    Jag vill ha mer jul
    Ge mig mer jul
    Jag vill ha mer jul
    Ge mig mer jul
    Tusen stjärnor som tindrar,
    glitter så långt jag ser
    Av juleljus som glimmar,
    vill jag ha mer


\end{parse lines}

\vissteduatt{Visste du att .... }

\newpage

\begin{parse lines}[\noindent]{#1\\}
    En show glöms bort om den bara visar opp 
    effekter som man knappast anar
    Så ge mig trettio grader kallt, tomtar överallt 
    och en skog av gröna granar
    Jag vill ha snötyngda hus, tusentals ljus, 
    kulörta kulor i drivor
    Bjällerklang som ackompanjemang 
    på alla julens skivor

    Jag vill ha mer jul…

    Ge mig en svårknäckt nöt, sötare gröt, 
    djupare dopp i grytan
    Glittrigare glim och grötigare rim 
    och mer Arne Weise i rutan
    Jag vill ha rymligare säck, segare knäck, 
    fetare fläsk från grisen
    Krimsigare krams, längre långdans 
    och raskare räv på isen

    Jag vill ha mer jul…

    Jag vill ha mer, mer
    Ge mig mer, mer
    Jag vill ha mer, mer
    Ge mig mer, mer

\end{parse lines}

\vissteduatt{Visste du att ....}

\newpage


\input{kapitel/03-andra_sektioner.tex}

\input{kapitel/04-klassiska_visor.tex}

\begin{center}
    \vspace*{1.5cm}
    {\fontsize{20}{20}\textbf{\fontspec{Lucida Sans Unicode}Skånska visor}}\\
    \vspace{0.7cm}
    {\fontsize{12}{12}\textit{Om gåsapågen själv får välja}}
\end{center}
\addtocwithheader{Skånska visor}  % Add entry to TOC and set header\noBackground
\noBackground

\newpage
\resetBackground



\subsection*{Skånsk ordlista}

Om du inte kommer från Skåne och nyss flyttat hit för att plugga på LTH, 
kan det ibland vara svårt att förstå alla ord som sägs. Då kan det vara bra
med en skånsk ordlista. \\



Mög - Skit

Rullebör - Skottkärra

Hutta - Kasta

Räligt - Äckligt

Hem om - Att åka förbi hem innan man åker någonstans

Fubbick - Idiot

Hialös - Jäktad, rastlös

Påg - Pojke

Tös, tösabid - Flicka

Päror - Potatis

Glytting - Barnslig

Klyddigt - Besvärligt

Klydderöv - Krånglig person

Skula - Att ta skydd mot regn

Prega - Att pressa/trycka

Sulten - Mycket hungrig

Läbbigt - Läskigt

Redig - Ordentlig

Blannevann - Groggvirke

Flabben - Käften/munnen

Hossor - Strumpor

Ålahue - Korkad person

Puligt - Najs

\newpage


\subsection*{Vi klarar oss nog ändå} 
\index[alfa]{Vi klarar oss nog ändå}
\index[anfa]{Jag vill sjunga en visa i klaraste dur}
\songinfo{Text: Lasse Dahlquist\\Sång: Edvard Persson}

\begin{parse lines}[\noindent]{#1\\}
    Jag vill sjunga en visa i klaraste dur,
    ty den handlar om Skåne och slätter och djur
    Kan hända den retar en del 
    men i så fall är det deras eget fel
    Det har talats så mycket om dynga och lort 
    men betänk vilken oerhörd nytta den gjort
    ||: Så låt dem bara gå på, vi klarar oss nog ändå :||
    % Ja låt dom bara gå på vi klarar oss nog ändå

    Kanske språket vi talar ej klingar så väl,
    men det är och förbliver en del av vår själ
    Kan hända det retar en del,
    men i så fall är det deras eget fel
    Uti självaste riksda'n på skånska dom slåss,
    för de flesta utav dom har kommit från oss
    ||: Så låt dem... :||
    % Så låt dom bara gå på...

    Utav våra produkter de smörjer sitt krås,
    och de är ifrån oss som dom fått Mårten Gås
    Kan hända det retar en del,
    men i så fall är det deras eget fel
    Det har klagats på bostaden på våra svin
    men när julskinkan kommer jo då är den fin
    ||: Så låt dem... :||
    % Så låt dom bara gå på...
\end{parse lines}

\newpage
\noBackground
% \begin{textblock*}{3cm}(4.0cm,8.0cm) % {width}(x, y)
%     \includegraphics[width=6.5cm]{./bilder/majas-bilder/nils_holgersson.png}
% \end{textblock*}

\begin{textblock*}{3cm}(0.4cm,7.5cm) % {width}(x, y)
    \includegraphics[width=10cm]{./bilder/majas-bilder/goose_utan_text.png}
\end{textblock*}

\begin{parse lines}[\noindent]{#1\\}
    Hela landet får njuta av vår akvavit
    Sockerbetan har lärt dom att dricka på bit
    Kan hända det retar en del,
    men i så fall är det deras eget fel
    Våran sandstrand den är både bländvit och fin
    och så har vi ju vår lilla vita kanin
    ||: Så låt dem... :||
    % Så låt dom bara gå på...

    Selma Lagerlöv som är en fin gammal dam,
    med Nils Holgersson gjorde för Skåne reklam
    Kan hända det retar en del,
    men i så fall är det deras eget fel
    % Kan hända det retar…
    Tänk sån nytta som storken från Skåne har gjort,
    men det hindrar ju inte att folk pratar lort
    % Men låt dem bara gå på vi klarar oss nog ändå
    ||: Så låt dem... :||
\end{parse lines}

\vissteduatt{Visste du att denna visa med fördel sjunges på grov skånska?}

\newpage
\resetBackground

\subsection*{Landskrona} 
\index[alfa]{Landskrona}
\index[anfa]{I Landskrona}
\songinfo{Mel: In the Ghetto}

\begin{parse lines}[\noindent]{#1\\}
    På en regnig strand
    längs Skånes gråa västra kust 
    bor några tusen fast dom inte har lust
    I Landskrona
    (I Landskrona)
    
    Flytta gärna dit
    Våran utsikt den är ganska käck
    om man bortser ifrån Barsebäck
    Och det gör man
    (Inte riktigt)
    
    På krogen varje fredagskväll
    där hänger Per och Kjäll
    Två fruktade lynniga bröder ifrån Ven
    
    Det är en kille ifrån Höör som stör
    han har snott deras rullebör
    Nu ska dom ge han vad han tål 
    med rör av rostfritt stål
    
    Sicket tufft klimat
    Men det finns en sak som förenar oss 
    det är stadens stolthet BoIS förstås
    Heja svartvitt
    (Heja svartvitt)
\end{parse lines}

\newpage

\begin{parse lines}[\noindent]{#1\\}
    Att spela boll är kul
    På Landskrona IP står ett randigt gäng
    dom har än en gång fått rejält med däng
    I Landskrona
    (Utav Mjellby)
    
    Sånna dagar känns det skit att va landskronit
    Vill ta pågatåget bort från stan
    men på just den dan kan man ge sig fan
    på att det är strejk
    
    Så man vandrar hem en kulen kväll
    och man träffar Kjell som åkt på en smäll
    Så han blör
    (Jäkla Höör)
    
    Men de ger aldrig upp
    för nästa helg tar vi nya tag
    Både BoIS och Per och Kjell och jag
    I Landskrona
    (I Landskrona)
\end{parse lines}

\newpage

\subsection*{Lite grann från ovan} 
\index[alfa]{Lite grann från ovan}
\index[anfa]{Jag är en liten gåsapåg från Skåne}
\songinfo{Text: Lasse Dahlqvist\\
Sång: Edvard Persson}

\begin{parse lines}[\noindent]{#1\\}
    Jag är en liten gåsapåg från Skåne
    en skåning som ni vet är alltid trygg
    Och fast jag är så nära sol och måne
    jag sitter säkert på min gåsarygg
    Långt under mig det ligger som en tavla
    det vackraste i världen man kan se
    Både skogar, sjö och strand
    blir ett enda sagoland
    när man ser det lite grann så här från ovan

    Där ligger gamla slott och herresäten
    som minnen från den stolta tid som flytt
    Och aldrig skall den tiden bli förgäten
    men inget slag skall stånda här på nytt
    Nej, dessa fält skall bära samma skördar
    som de har gjort i sekelflydda dar
    Ja det är min liv och kniv
    alla tiders perspektiv
    när man ser det lite grann så här från ovan
\end{parse lines}

\vissteduatt{Visste du att 2020 genomfördes nollEgasque via Zoom till följd\\
 av pandemin? E6 körde ut maten med bil.}

\newpage
\noBackground

\begin{textblock*}{3cm}(3.0cm,7.0cm) % {width}(x, y)
    \includegraphics[width=7.5cm]{./bilder/majas-bilder/nils_holgersson.png}
\end{textblock*}

\begin{parse lines}[\noindent]{#1\\}
    Du kära gås som stolt i skyn dig svingar
    har ingen farlig last att kasta ned
    Ty du bär fredens vita vackra vingar
    som världen längtar efter mer och mer
    När människobarnen går därnere och kivas
    då resonerar du nog liksom jag
    Tänk vad skönt det är ändå
    att få sväva i det blå
    och se er lite grann så här från ovan

    Nu jordens alla murar börjar skaka
    en samling dårar satt vår värld i brann
    Vad fäderna byggt upp blir pannekaka
    Det ryck och slits i gamla vänskapsband
    Ack kära ni som slåss därner på jorden
    kom upp och ta en liten titt med mig
    Jag är ganska säker på
    att ni skäms en smula då
    när ni ser vår gamla jord så här från ovan
\end{parse lines}

\vissteduatt{Visste du att 2021 genomfördes nollEgasque två gånger på \\
samma dag till följd av pandemin?}

\newpage 
\resetBackground

\subsection*{Skånsk madavisa} 
\index[alfa]{Skånsk madavisa}
\index[anfa]{Rabbemos å spegefläsk}
\songinfo{Mel: Aspelöv ock Lindelöv}

\begin{parse lines}[\noindent]{#1\\} 
Rabbemos å spegefläsk,
panntofflagröd med knudor,
fläskasvålar, grisatassar,
prinsakorv med snudor
Fittamadar, sillarumpor,
sylta med rödbedor,
äggakaga, revbensspjäll

Luad ål å rögad ål
å ål som di har halmad
Kogad ål å stegad ål
å ål i gelatin
Ålasluring, ålapudding
ål med chokeladsås,
rutten ål å ål i kål

Hussegröd å puggavälling,
kläggefläsk med bläror
Tösaflabbar, flinerumpor,
pattagris med päror
Glyttanäsor, hunnarövar,
lummemög med hylle,
mormor Karnas hönsafjös

Sillasupen, ålasupen,
supen till sardellen
Fläskasupen, rabbesupen,
suparna till spjällen
Gåsasupen, äggasupen,
suparna till supen
å till sist en kagesup

Spiddekaga, kransekaga,
flarn å mazariner,
sockerkaga, butterkaga
nötter å praliner
Risengröd med vispegrädde,
punsch å karameller
Sen e de' dags för nattamad!

\end{parse lines}

\subsection*{Eslövs nationalsång} 
\index[alfa]{Eslövs nationalsång}
\index[anfa]{Vi går här på slätten och vi hackar våra bedor}

\begin{parse lines}[\noindent]{#1\\}

    ||: Vi går här på slätten
    och vi hackar våra bedor
    Vi går här på slätten och hackar hela dan :||
    ||: Åååh, hackar våra bedor
    Åååh, hackar hela dan :||

    ||: Jag ska vända mig, och bocka mig,
    och ta en liten beda
    Jag ska vända mig, och bocka mig,
    och ta en beda till :||
    ||: Åååh, jag tar en liten beda
    Åååh, jag tar en beda till :||
\end{parse lines}

\vissteduatt{Vissde due add skeuåska aer ded feijnaste spraåked i hela \\
vaerlden?}
\newpage

\subsection*{Skåne} 
\index[alfa]{Skåne}
\index[anfa]{Nu har det blivit dags att dricka Skåne}
\songinfo{Mel: Lite grann från ovan}

\begin{parse lines}[\noindent]{#1\\}
    Nu har det blivit dags att dricka Skåne
    en liten sträv, så gyllengul som raps
    Det är det bästa mellan sol och måne
    som lätt får mången stark till snabb kollaps
    Så fatta nu din hand om hela Skåne
    här kommer södra Sverige i en snaps
    Känn hur Lund och Smygehuk
    slår små volter i din buk
    Aquaviten ska va' gul
    och heta Skåne
\end{parse lines}


\subsection*{Till den skånska metropolen Vinslöv} 
\index[alfa]{Till den skånska metropolen Vinslöv}
\index[anfa]{I vin, i vin, i vin, i vin}
\songinfo{Mel: Wiensk operettvals}

\begin{parse lines}[\noindent]{#1\\}
    I vin, i vin, i vin, i vin,
    i Vinslöv bor min mor
    På Hven, på Hven, på Hven, på Hven,
    påven han bor i Rom
    I Rom, i Rom, i Rom, i Rom,
    i rompan på en ko
    I ko, i ko, i ko, i ko,
    i Kosta göra man glas
    I glas, i glas, i glas, i glas,
    i glas där har man vin
    I vin, i vin, i vin, i vin,
    i Vinslöv bor min mor
\end{parse lines}

\vissteduatt{Visste du att fler snapsvisor finns på sida 108?}
\newpage
%\vspace*{-4cm}
\enlargethispage{5cm}
%\vspace*{3.6cm}
\subsection*{Skånska slott och herresäten} 
\index[alfa]{Skånska slott och herresäten}
\index[anfa]{På himmelen vandra sol, stjärnor och måne}
\songinfo{Text: Hjalmar Gullberg \& Bengt Hjelmqvist\\
Sång: Edvard Persson}

\begin{parse lines}[\noindent]{#1\\}
    På himmelen vandra sol, stjärnor och måne
    och kasta sitt fagraste ljus över Skåne
    På höga och låga, på stort och på smått
    på statarens koja och ädlingens slott

    Se månstrålen in genom blyrutan faller
    och tecknar på golvet det järnsmidda galler
    Skön jungfrun hon drömmer i majnattens ljus
    att friare komma till Glimmingehus

    På utflykt till bokskogen malmöbon glor upp
    mot raden av strålande fönster på Torup
    Att smaka på kaka som bakats på spett
    dig ber hennes nåd, friherrinnan Coyet

    Där rådjuren skymta bak'vitgråa stammar
    man ser Toppela'gård med broar och dammar
    Långt bort från Systemet och spriten och sta'n
    där lurar belåtet fiskalen Aschan

    Med port genom huset och gamla kanaler
    lyss Skabersjö ännu till jaktens signaler
    Själv kungen i nåder far dit från sitt slott
    och skjuter fasaner med grevarna Thott

    Och därefter hälsar han på baron Trolle
    och jagar och spelar sin sans och sin nolle
    Allt medan baronens gemål
    plockar gräs åt rastupp och rashöna på Trollenäs
\end{parse lines}
\newpage

\begin{center}
    \vspace*{1.5cm}
    {\fontsize{20}{20}\textbf{Teknologvisor}}\\
    \vspace{0.7cm}
    {\fontsize{12}{12}\textit{Om teknologen själv får välja}}
  
    \end{center}
    \noBackground

    \newpage
    \resetBackground


\subsection*{Porthos visa} 
\index[alfa]{Porthos visa}
\index[anfa]{Jag vill börja gasqua!}
\songinfo{Mel: You can't get a man with a gun (ur Annie get your gun)}

\begin{parse lines}[\noindent]{#1\\}
    Jag vill börja gasqua!
    Var fan är min flaska?
    Vem i helvete stal min butelj?
    Skall mej törsta betvinga?
    En TT börja svinga?
    Nej för fan bara blunda och svälj!
    Vilken smörja!
    Får jag spörja?
    Vem för fan tror att jag är en älg?
    Till England vi rider,
    och sedan vad det lider,
    träffar vi välan på någon PUB
    Och där skall vi fest!
    Blott dricka utav det bästa
    utav Whiskey och Portvin
    Jag tänker gå hårt in
    för att prova på rubb och stubb
    Rubb och stubb…
\end{parse lines}

\vissteduatt{Visste du att om den sista "stubb" sjunges så måste låten upprepas...}

% \newpage


\subsection*{En komplex värld} 
\index[alfa]{En komplex värld}
\index[anfa]{Alla jävla bevis}
\songinfo{Mel: En helt ny värld (ur Aladdin)\\
Text: Ellinor Persson F07 och Andreas Tågerud F06
}

\begin{parse lines}[\noindent]{#1\\}
    Alla jävla bevis
    inses lätt som en övning.
    Javakursen en prövning
    för min bristande logik.

    Ska man komma ihåg
    alla formler i huvet?
    Formelsamlingen, du vet,
    säger inget om det här!

    En komplex värld
    Vad fan betyder bijektiv?
    Ingenting stämmer här, där allt jag lär,
    blir glömt snart efter tentan.
    Hur ska det gå?
    Och det är bara vecka två...
    Känner en underton av aggression
    mot allt Sven Spanne skrivit i sin bok

    (Jag kan transponera den...)
\end{parse lines}

\newpage

\begin{parse lines}[\noindent]{#1\\}
    Jag kan lära dig C
    Matematiska under
    Oförglömliga stunder
    när vi tentar mekanik

    Det ska nog gå!
    Det sa din mamma med igår
    All tid tillvaratas, jag är i fas,
    och bor i mattehuset.
    Nu är jag lärd!
    Till denna svåra ekvation
    jag på frekvenssidan en lösning fann,
    den låg där i en helt ny värld: Laplace!
\end{parse lines}


\subsection*{Enhetsvisan / SI - Système International d'Unités} 
\index[alfa]{Enhetsvisan / SI - Système International d'Unités}
\index[anfa]{1, 2, 75, 6, 7}
\songinfo{Mel: Studentsången}

\begin{parse lines}[\noindent]{#1\\}
    W Kg m Wb s
    Ω m T A rad
    cd Sv N s
    Ω A m lx dB
    °C W/m²
    J/kg H V C
    kg/m3 mol
    m/s²
    m/s²
    F !
\end{parse lines}

\vissteduatt{Denna finns i TEfyma}

\newpage


\subsection*{Man ska ha matlab} 
\index[alfa]{Man ska ha matlab}
\index[anfa]{Man ska ha matlab}
\songinfo{Mel: Man ska ha husvagn}

\begin{parse lines}[\noindent]{#1\\}
    Jag har prövat nästan allt som finns att pröva på
    Beta, kulram, räknesticka, tärning eller så
    Jag har kalkylerat på de konstigaste sätt
    och nu så har jag kommit på hur man ska räkna rätt

    Man ska ha MATLAB - då är kalkylen redan klar
    Man ska ha MATLAB - det har jag sett att andra har
    Man ska ha MATLAB - det är min livsfilosofi
    Man ska ha MATLAB - för då blir man fri

    I många år så var jag inte alls så särskilt lärd
    Jag visste ej vad som vänta' mig i denna stora värld
    Men sen kom jag till LTH, och ända sedan dess
    så har jag funnit livets stora lyxdelikatess

    Man ska ha MATLAB - så man slipper tänka alls
    Man ska ha MATLAB - ja, då går allting som en vals
    Man ska ha MATLAB - det bygger på nån slags logik
    Man ska ha MATLAB - för då blir man rik

    5 minuter mekanik och 5 minuter statfys
    5 minuter plottande och 5 minuter analys
    5 minuter fråga phadder, 5 minuter stopp
    5 minuter tänka själv och sen så ger man opp
\end{parse lines}

\newpage

\begin{parse lines}[\noindent]{#1\\}
    Man ska ha MATLAB - och datasalens friska luft
    Man ska ha MATLAB - det tycker tjejerna är tufft
    Man ska ha MATLAB - när ryssen kommer med sitt MIG
    Man ska ha MATLAB - då vinner man i krig!
\end{parse lines}

\subsection*{Tar bort: Atomvinter} 
% \index[alfa]{Man ska ha matlab}
% \index[anfa]{Man ska ha matlab}
% \songinfo{Mel: Man ska ha husvagn}

\begin{parse lines}[\noindent]{#1\\}
    Tas bort
\end{parse lines}

\vissteduatt{visste du att}

\newpage


\subsection*{Teknologvisa} 
\index[alfa]{Teknologvisa}
\index[anfa]{Jag är teknolog och helt OK}
\songinfo{Mel: I’m a lumberjack (Monty Python)\\ Sångarstriden 1982\\ Kursivt sjunges av sångförman}

\noindent\textit{Jag är teknolog och helt OK\\
Jag jobbar hårt och jag roar mig}\\\\
\noindent Han är teknolog och helt OK\\
Han jobbar hårt och han roar sig\\\\
\noindent\textit{Teknik är ball\\
Jag kan Pascal\\
Till Lophtet vill jag gå\\
Där träffas alla vänner\\
som är från LTH}\\\\
\noindent Teknik är ball\\
Han kan Pascal\\
Till Lophtet vill han gå\\
Där träffas alla vänner\\
som är från LTH\\\\
\noindent För han är teknolog och helt OK\\
Han jobbar hårt och han roar sig\\\\


\vissteduatt{Visste du att "Han" kan eneklt bytas ut mot "Hon" eller "Hen"?}
\newpage


\noindent\textit{Min mattebok \\
den gör mig klok\\
Jag läser kärnfysik\\
Jag går på föreläsning\\
och älskar juridik}\\\\
\noindent Hans mattebok\\
den gör han klok\\
Han läser kärnfysik\\
Han går på föreläsning\\
och älskar juridik???\\\\
\noindent Men han är teknolog och helt OK\\
Han jobbar hårt och han roar sig\\\\
\noindent\textit{Som ekonom jag blir fantom\\
Konkurser gör mig säll\\
Till flickor blankt jag nekar\\
Jag älskar en tabell}\\\\
\noindent Som ekonom han blir fantom???\\
konkurser...\\
Nää, BUU!!\\\\
\noindent Men han är teknolog och helt OK\\
Han jobbar hårt och han roar sig\\

\vissteduatt{visste du att}

\newpage




\begin{center}
    \vspace*{1.5cm}
    {\fontsize{20}{20}\textbf{Ölvisor}}\\
    \vspace{0.7cm}
    {\fontsize{12}{12}\textit{Om jästen själv får välja}}
\end{center}
\addcontentsline{toc}{section}{Ölvisor}
\noBackground

\newpage
\resetBackground


\subsection*{Ode till ölet} 
\index[alfa]{Ode till ölet}
\index[anfa]{Tu tu tu Tuborg}
\songinfo{Mel: Trampa på gasen}

\begin{parse lines}[\noindent]{#1\\}
    Tu tu tu Tuborg
    och ca ca ca Carlsberg
    det är den bästa
    pi pi pi pilsnern som jag vet

    Tu tu tu Carlsberg
    och ca ca ca Tuborg
    det är det bästa
    pi pi pi ölet som jag vet

    Tu tu tu Ölberg
    och ca ca ca Pilsborg
    det är den bästa
    pi pi pi biran som jag vet

    Tu ca pi Ölsner
    och pi tu ca bira
    det är den bästa
    ca pi tu lering som jag gjort
\end{parse lines}

\vissteduatt{Visste du att...}

\newpage

\subsection*{Om en söt dryck} 
\index[alfa]{Om en söt dryck}
\index[anfa]{Man é ej dum för att man dricker cider}
\songinfo{Mel: Jag fångade en räv idag\\
Text: Erik Johanneson F99}

\begin{parse lines}[\noindent]{#1\\}
    Jag dricker gärna öl och vin
    till sillen tar jag nubben
    och kanske till och med en shot
    när jag har gått till klubben.

    Men drycken som jag helst vill ha
    den har så friska lukter
    den är dessutom ren och klar
    och smakar utav frukter

    Hå hum, man é ej dum
    för att man dricker cider
    Även om, det finnes dom
    som utav smaken lider

    Det skummar upp i glaset när
    man har hällt upp en cider
    det smakar både sött och gott
    när den i halsen glider
\end{parse lines}


\vissteduatt{Visste du att...}
\newpage

\begin{parse lines}[\noindent]{#1\\}
    Det kan vá både söt och torr
    av tranbär eller fläder
    av plommon och det finns nån sort
    som smakar gamla kläder

    Hå hum, man é ej dum...
\end{parse lines}

\subsection*{\colorbox{orange}{Ölvisa från Veberöd}} 
% \index[alfa]{Om en söt dryck}
% \index[anfa]{Man é ej dum för att man dricker cider}
\songinfo{Mel: I ett hus vid skogens slut\\
Lundakarnevalen 1994
}

\begin{parse lines}[\noindent]{#1\\}

    KANSKE SKA TAS BORT??????????????????????????????

    Ur ett glas vid bordets slut
    liter starköl rinner ut
    värden tittar på sin stop
    bryter snart ihop 

    Nej, han vaknar ur sin nöd
    cyklar hem till Veberöd
    pantsätter sin systers föl
    tusen spänn till öl!

\end{parse lines}

\vissteduatt{Visste du att...}

\newpage

\subsection*{\colorbox{red}{Guld}} 
% \index[alfa]{Om en söt dryck}
% \index[anfa]{Man é ej dum för att man dricker cider}
% \songinfo{Mel: I ett hus vid skogens slut\\
% Lundakarnevalen 1994
% }

\begin{parse lines}[\noindent]{#1\\}
    tas bort
\end{parse lines}


\subsection*{Ölkanon} 
\index[alfa]{Ölkanon}
\index[anfa]{Drick, drick, drick din öl}
\songinfo{Mel: Row your boat\\
Lundakarnevalen 2002}

\begin{parse lines}[\noindent]{#1\\}
    Drick, drick, drick din öl,
    låt den rinna ner.
    Kan du sen kraxa,
    “en laxask med slasktratt”
    så får du dricka fler
\end{parse lines}


\subsection*{Oralöl} 
\index[alfa]{Oralöl}
\index[anfa]{Min vän Kal, är Dual}
\songinfo{Mel: B.L.O.S.S.A\\
Lundakarnevalen 2006}

\begin{parse lines}[\noindent]{#1\\}
    Min vän Kal, är Dual
    Hans anal, är verbal
    Så när Kal, håller tal,
    kan han halsa pilsner!
\end{parse lines}

\vissteduatt{Visste du att...}


\newpage


\subsection*{Strejk på Pripps} 
\index[alfa]{Strejk på Pripps}
\index[anfa]{Inatt jag drömde något som...}
\songinfo{Mel: I natt jag drömde}

\begin{parse lines}[\noindent]{#1\\}
    Inatt jag drömde något som,
    jag aldrig drömt förut
    Jag drömde det var strejk på Pripps
    och alla ölen var slut
    Jag drömde om en jättesal
    där ölen stod på rad
    Jag drack sådär en femton öl
    och reste mig och sa:
    "Man kan ha roligt utan sprit,
    men det är dumt att chansa!"
\end{parse lines}


\subsection*{Ont i huvudet} 
\index[alfa]{Ont i huvudet}
\index[anfa]{Om du har ont i huvet}
\songinfo{Mel: Jag har en gammal faster}

\begin{parse lines}[\noindent]{#1\\}
    Om du har ont i huvet
    när du vaknar någon da'
    så häll en öl i håret
    och låt den stå och dra

    Och känns det inte bättre
    så skyll inte på oss,
    Att hälla öl i huvet
    är inte smart förstås
\end{parse lines}

\vissteduatt{Visste du att...}


\newpage


\subsection*{Var nöjd med ölen} 
\index[alfa]{Var nöjd med ölen}
\index[anfa]{Var nöjd med allt som ölen ger}
\songinfo{Mel: Var nöjd med allt vad livet ger}

\begin{parse lines}[\noindent]{#1\\}
    Var nöjd med allt som ölen ger
    och även om du dubbelt ser
    Glöm bort bekymmer sorger och besvär
    Var glad och nöjd för vet du vad
    en folköl gör ju ingen glad
    Var nöjd med ölen som vi dricker här
\end{parse lines}

\vissteduatt{Visste du att...}

\newpage


\subsection*{Öl, öl, öl i glas} 
\index[alfa]{Öl, öl, öl i glas}
\index[anfa]{Öl, öl, öl i glas}
\songinfo{Mel: Row your boat}

\begin{parse lines}[\noindent]{#1\\}
    Öl, öl, öl i glas eller i butelj
    Skummande, skummande,
    skummande, skummande
    Ta en klunk och svälj
\end{parse lines}

\vissteduatt{Visste du att...}


\newpage


\subsection*{\colorbox{red}{Tar bort: prostatabesvär}} 
% \index[alfa]{Öl, öl, öl i glas}
% \index[anfa]{Öl, öl, öl i glas}
% \songinfo{Mel: Row your boat}

\begin{parse lines}[\noindent]{#1\\}
    tas bort
\end{parse lines}

\vissteduatt{Visste du att...}


\newpage


\subsection*{Min pilsner} 
\index[alfa]{Min pilsner}
\index[anfa]{Min pilsner}
\songinfo{Mel: My Bonnie}

\begin{parse lines}[\noindent]{#1\\}
    Min pilsner skall svalka min tunga
    Min pilsner skall duscha min gom
    Min pilsner skall få mig att sjunga,
    om jag ser att flaskan är tom:
    Pilsner! Pilsner!
    Hämta en pilsner till mig, till mig
    Pilsner! Pilsner!
    Hämta en pilsner till mig!
\end{parse lines}


\subsection*{Min trasa} 
\index[alfa]{Min trasa}
\index[anfa]{Min trasa}
\songinfo{Mel: My Bonnie\\
Text: Kewin Erichsen E11, Henrik Fryklund E13}

\begin{parse lines}[\noindent]{#1\\}
    Min pilsner har spillts ut på bordet
    Min pilsner har lämnat mitt glas
    Min pilsner har sölat min ovve
    och nu blir grannen indignerad på mig

    Trasa! Trasa!
    Hämta en trasa till mig, till mig
    Trasa! Trasa!
    Hämta en ny pilsner till mig!
\end{parse lines}

\vissteduatt{Visste du att Min Trasa skrevs för att fylla ut sidan?}


\newpage


\subsection*{Då ölen är kall} 
\index[alfa]{Då ölen är kall}
\index[anfa]{Det är nu som ölen är kall}
\songinfo{Mel: Gabriellas sång\\ Text: Axel Lundholm E12, Hugo Hjertén E12}


\noindent Det är nu som ölen är kall, \textbf{bam bam}\\
\begin{parse lines}[\noindent]{#1\\}
    Den har väntat på mig i kylen
    Sakteligen jag rör mig dit
    För att stilla min bittra törst
\end{parse lines}
\noindent Tidigare samma dag, \textbf{bam bam}\\
\begin{parse lines}[\noindent]{#1\\}
    Jag begav mig mot systemet
    Jag var där strax innan tre
    Men systemet det stängde två!
    
    Jag vill hälla öl i strupen
    Humle mot min gom
    Men kylen den står tom
    Ölen som jag hoppats på 
    Är blott en fakking illusion
    
    …Men!
    Jag har faktiskt ännu än öl i min kyl!
\end{parse lines}


\vissteduatt{Visste du att\dots}

\newpage

\begin{center}
    \vspace*{1.5cm}
    {\fontsize{20}{20}\textbf{\fontspec{Lucida Sans Unicode}Vinvisor}}\\
    \vspace{0.7cm}
    {\fontsize{12}{12}\textit{Om den vita skjortan själv får välja}}
\end{center}
\addtocwithheader{Vinvisor}  % Add entry to TOC and set header\noBackground
\noBackground

\newpage
\noBackground

\begin{textblock*}{3cm}(4.1cm,7.3cm) % {width}(x, y)
    \includegraphics[width=6cm]{./bilder/majas-bilder/feta_fransyskor.png}
\end{textblock*}


\subsection*{Feta fransyskor} 
\index[alfa]{Feta fransyskor}
\index[anfa]{Feta fransyskor som svettas om fötterna}
\songinfo{Mel: Militärmarsch av Schubert\\
K-sektionen Sångarstriden 1985}

\begin{parse lines}[\noindent]{#1\\}
    Feta fransyskor som svettas om fötterna,
    de trampar druvor
    som sedan ska jäsas till vin
    Transpirationen viktig é,
    ty den ge'
    fin bouquet
    Vårtor och svampar följer me'
    men vad gör väl de'?

    För vi vill ha vin,
    vill ha vin,
    vill ha mera vin,
    även om följderna blir
    att vi må lida pin
    Flaskan och glaset gått i sin
    Hit med vin, mera vin!
    Tror ni att vi är fyllesvin?
    {\Large Ja!} (Fast större!)
\end{parse lines}


\newpage
\resetBackground

\subsection*{Imsig vimsig} 
\index[alfa]{Imsig vimsig}
\index[anfa]{xsnan@Åsnan dricker vatten}
\songinfo{Mel: Imse vimse spindel}

\begin{parse lines}[\noindent]{#1\\}
    Åsnan dricker vatten, det gör inte vi
    Vi dricker bara sådant folk har trampat i
    Kamelen uti öknen söker en oas
    Det gör inte vi, vi har vin i våra glas

    Imsig vimsig blir man, utav lite vin
    Kliver upp på stolen, verkar piggelin
    Ramlar under bordet, sussar en minut
    Vaknar av att vinet i glaset tagit slut

\end{parse lines}


\subsection*{Korkskruvens visa} 
\index[alfa]{Korkskruvens visa}
\index[anfa]{Nu har vi rus här i vårt hus}
\songinfo{Mel: Nu har vi ljus}

\begin{parse lines}[\noindent]{#1\\}
    Nu har vi rus här i vårt hus
    Korken är borta hopp tralalala
    Doften är ljuv, jag är en skruv,
    jag är en skruv
    Jag kan inte öppna bag-in-boxen
    Jag kan inte öppna bag-in-boxen
    Lalala lalalalala 
    lalalalala lala lala
\end{parse lines}

\vissteduatt{Visste du att radioentusiaster från ETF lade grunden till det som \\idag är Radio AF?}

\newpage


\subsection*{Lyft ditt välförsedda glas} 
\index[alfa]{Lyft ditt välförsedda glas}
\index[anfa]{Lyft ditt välförsedda glas}
\songinfo{Mel: Ding Dong Merrily on High}

\begin{parse lines}[\noindent]{#1\\}
    Lyft ditt välförsedda glas
    Det är en härlig börda
    Nu har grabbarna kalas
    Vi segern snart skall skörda
    ||: Ding dingedingeding dingedingeding
    dingedingeding dong dong
    Imorgon är det lördag :||

    Sätt nu glaset till din mun
    Se döden på dig väntar
    Nu har grabbarna kalas
    Hör liemannen flämtar
    ||: Ding dingedingeding dingedingeding
    dingedingeding dong dong
    Begravningsklockor klämtar :||

\end{parse lines}

\newpage

\hspace{1cm}
\subsection*{Bordeaux, bordeaux} 
\index[alfa]{Bordeaux, bordeaux}
\index[anfa]{Jag minns än idag hur min fader}
\songinfo{Mel: I sommarens soliga dagar}

\begin{parse lines}[\noindent]{#1\\}
    Jag minns än idag hur min fader
    kom hem ifrån staden så glader
    och rada' upp flaskor i rader
    och sade nöjd som så:
    "Bordeaux, Bordeaux!"

    Han drack ett glas, kom i extas,
    och sedan blev det stort kalas
    Och vi små glin, ja vi drack vin
    som första klassens fyllesvin
    Och vi dansade runt där på borden
    och skrek så vi blev blå:
    "Bordeaux, Bordeaux!"


\end{parse lines}

\subsection*{Vinbröder} 
\index[alfa]{Vinbröder}
\index[anfa]{Två bröder, Jan-Ove och Hadar}
\songinfo{Mel: I sommarens soliga dagar}

\begin{parse lines}[\noindent]{#1\\}
    Två bröder, Jan-Ove och Hadar,
    de plockade fram sina spadar,
    och grävde en grop bakom huset.
    De hade en idé:
    Vadå, vadå?
    


    Jo, priset på, Kir och Bordeaux,
    är högt men om man gjorde så,
    att man i gropen lade ner,
    två kilo jäst och en back MER,
    så skulle det nog vara möjligt
    att producera vin i detta hål!


\end{parse lines}


\subsection*{Karnaugh, karnaugh} 
\index[alfa]{Karnaugh, karnaugh}
\index[anfa]{Karnaugh, karnaugh}
\songinfo{Mel: I sommarens soliga dagar}

\begin{parse lines}[\noindent]{#1\\}
    Jag minns än i dag hur min fader
    kom hem i från labbet så glader
    och rada' upp bitar i rader
    och sade glad som så:
    Karnaugh, Karnaugh!

    Å ett å noll \& noll å ett 
    å booleska uttryck det är fett!
    Med ett å noll \& noll å ett,
    ska du nu se att det blir rätt,
    Men vi felsökte våra signaler,
    och det blev fel ändå
    Karnaugh, Karnaugh!
\end{parse lines}


\newpage


\subsection*{I sommarens soliga dagar} 
\index[alfa]{I sommarens soliga dagar}
\index[anfa]{I sommarens soliga dagar}
\songinfo{Mel: I sommarens soliga dagar\\
Text: Anders Nilsson $\pi$03 \& Björn Carlin $\pi$02}

\begin{parse lines}[\noindent]{#1\\}
    I sommarens soliga dagar
    Kall rå fisk till alla vi lagar
    Och sätter oss ute i hagar
    Där maten härsken står

    Bakteriehärd av solen närd
    Den höjer högt sitt blanka svärd
    Men mot dess hot vi funnit bot 
    Vi sätter spritens krafter mot
    För spriten kan döda det mesta
    Vi hälsan återfår 
    Gutår! Gutår!
\end{parse lines}

\vissteduatt{Visste du att LTH-fontänen invigdes 1970 men plockades ner 1996
\\efter otaliga försök att reparera den? Kvar står stålskelettet.}

\newpage
\noBackground

\begin{textblock*}{3cm}(5cm,8cm) % {width}(x, y)
    \includegraphics[width=4cm]{./bilder/majas-bilder/vindruvor.png}
\end{textblock*}



\subsection*{Magnumflaskan Åkesson} 
\index[alfa]{Magnumflaskan Åkesson}
\index[anfa]{Magnumflaskan Åkesson}
\songinfo{Mel: Teddybjörnen Fredriksson\\
Lundakarnevalen 2010}

\begin{parse lines}[\noindent]{#1\\}
    För längesen, när jag fyllde 15 år
    fick jag en flaska av min mor
    Hon sa "Mitt barn, dela den med vännerna
    För den är faktiskt ganska stor."

    Magnumflaskan Åkesson,
    du var så stor och tung
    Jag gick runt med dig i hand,
    och i Dalby var jag kung

    Magnumflaskan Åkesson,
    din kork försvann i skyn
    Men jag drack dig bara själv
    Och kräktes i hela byn
\end{parse lines}


\newpage
\resetBackground

\subsection*{Detta är jag} 
\index[alfa]{Detta är jag}
\index[anfa]{Jag är en enkel mytoman}
\songinfo{Mel: Med en enkel tulipan\\
Lundakarnevalen 2018}

\begin{parse lines}[\noindent]{#1\\}
    Jag är en enkel mytoman
    Som utav ren slentrian
    Han ljugit till mig en plats vid bordet
    och ett glas rödvin    
    
\end{parse lines}

\subsection*{Kosmetisk visa} 
\index[alfa]{Kosmetisk visa}
\index[anfa]{Du behöver inte ha nåt läppstift alls}
\songinfo{Mel: She’ll be Coming ‘Round the Mountain\\
Lundakarnevalen 2002}

\begin{parse lines}[\noindent]{#1\\}
    Du behöver inte ha nåt läppstift alls,
    du behöver inte ha nåt läppstift alls,
    för när vinet börjar flöda,
    färgas läpparna ju röda,
    liksom tungan, hakan, skjortan och din hals!
\end{parse lines}

% \vissteduattlong{Visste du att det fanns en växlingsmojt i Edekvata inte bara kunde \\
% växla en tjuga mot rätt antal mynt, utan även hade en jackpott-funktion \\
% som slumpade ut enligt en normalfördelningskurva så att man ibland \\
% kunde få 23kr men andra gånger bara 18kr i mynt?}
\vissteduattlong{Visste du att det fanns en mojt i Edekvata som växlade tjugolappar \\till mynt? Mojten hade en jackpott-funktion som normalfördelat \\slumpade så att man ibland fick 23 kr, ibland endast 18 kr.}


\newpage
\noBackground


\begin{textblock*}{3cm}(5.8cm,3.8cm) % {width}(x, y)
    \includegraphics[width=4.0cm]{./bilder/dance_macabre_transparent.png}
\end{textblock*}

\subsection*{Dance macabre} 
\index[alfa]{Dance macabre}
\index[anfa]{runt våran stuga, små djävlar sluga}
\songinfo{Mel: Vårvindar friska}

\begin{parse lines}[\noindent]{#1\\}
    Runt våran stuga, små djävlar sluga
    tassa så tyst med bockfot och svans
    Varulvar yla, isande kyla
    sveper i dimman, fantygets dans
    Bäva, o broder, lyssna och hör
    vrålen från gast, som osalig dör
    Satan han skrattar,
    flaskan han fattar,
    super tills dagen gryr.

    Gastar och spöken
    skymta i köken,
    dödingar släpa ruttnande lik
    Benrangel skramla,
    spökhänder famla,
    kväva din strupes rosslande skrik
    Helvetes alla fasor släpps loss
    Fan rider här med hela sin tross
    Göm dig i stugan,
    du har fått flugan
    Dille det blir din lott

\end{parse lines}

\vissteduatt{Visste du att man viskar hela låten fram till "Helvetets alla \\
fasor..." då tar man i för Kung och fosterland?}


\newpage


\newpage
\resetBackground
\printindex[alfa]
\printindex[anfa]
\newpage

\end{document}
